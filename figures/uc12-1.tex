{ \tiny
	\begin{longtable}{|p{2cm} p{11cm}|}
		\hline
			\textbf{ID} & 
				\href{https://trello.com/c/2V6Cl65u}{UC12-1}\\
			
			\hline
			\textbf{Actor} & Statistician or Admin \\
			\hline
			\textbf{Description} & 
				Add Test-Argument into the system\\
			\hline
			\textbf{Desired~outcome} & 
				- A statistician can create new arguments \newline
				- These arguments can be assigned to models \newline
				- The arguments are accessible for all users in the system \newline
		\\
		\hline
			\textbf{Flow} & 
				1.) The user is logged in as Admin/Statistician  \newline
				2.) The User clicks on "Add Test-Argument" (see \href{https://trello.com/c/OwM2Z7wt}{Query arguments}, \href{https://trello.com/c/Rg6GPnNE/39-uc12-5-add-attacks-between-arguments}{Blank arguments})\newline
				3.) The system shows a form including "Name", "Description", "Attacking (Model or Other Argument)", "Required dataset fields", "R-Code to run test" and a disabled submit button\newline
				4.) The User enters a Name\newline
				5.) The user might enter a description\newline
				6.) The User selects at least on Model or other Argument from the list of possible attacked objects and whether it is critical or non-critical to that object.\newline
				7.) The user enters a list of required dataset fields.\newline
				8.) The user enters the R-Code to test if this argument holds or not. The R-Code must return a true/false statement at the end, \texttt{True} saying the argument holds, otherwise \texttt{False}. The R-Script will get as input a variable named \texttt{tabular\_data}  and must assign the variable \texttt{result} with \texttt{true} / \texttt{false} (see \autoref{sec:r_code}). The attributes specified in "Required dataset fields" will be checked before executing the \texttt{R}-Script.\newline
				9.) The user is in charge of verifying his script against datasets. A functionality to do so will be given by selecting a dataset in the system first. The script will then be checked against this dataset.\newline
				10.) The user clicks submit.The system performs a validation of the argument.\newline
				11.) The System stores the argument in the database and redirects the user to a list of available arguments and shows a success message.
		\\
		\hline
			\textbf{Alternatives} & 
							1.a) Not logged in as Admin: No changes to arguments possible.
				\newline	4.a) No name entered: show required inline message, disable submit button.
				\newline	6.a) No attacked object selected: show required inline message, disable submit button.
				\newline	8.a) If the user does not enter a R-Script: mark as required and do not enable submit button.
				\newline	9.a) If there is a syntax error: Show syntax error and line to the user
				\newline	9.b) If the result is not boolean: Show error explaining true/false result required.
				\newline	9.c) If there has been an error during execution with the testdataset: show error.
				\newline	9.d) After successful run: Show the dataset used to test it and the result.
				\newline	10.a) If the user does not select "Submit": NO changes done.
				\newline	11.a) If the validation fails: Redirect the user back to the form page and show validation failure.
				\newline	11.b) If any error occurs,  Redirect the user back to the form page and show error explanation.
							\\
		\hline
				\label{uc:12-1}\\
	\caption{Use case 12-1: Example for a detailed description}
	\end{longtable}
}