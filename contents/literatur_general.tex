\subsection{Argumentation Theory: General introduction}
\glsresetall
\glsunset{w.r.t.}
\label{sub:dung}
In the following section a general overview on \glspl{AF} will be given. The notation and definitions are based on Dung's theory \cite{dung1995} as it is a widely used definition for \glspl{AF} and the main sources of this dissertation \cite{Modgil2009,sassoon2016,sassoon2014,sassoon2016CD} are based on this approach.
\begin{definition}
	An argumentation framework is a tuple $AF = \langle \S, \R \rangle$ where $\S$ is a set of arguments and $ \R \subseteq \S \times \S$. $\R$ is a binary \textit{"attack relation"}.
\end{definition}

\begin{notation}
In this paper we use capital letters $\{A, B, ...\}$ to denote arguments. $AB$ or $(A, B)$ denotes an attack from $A$ to $B$ ($(A,B) \in \R$).	
\end{notation}

\begin{remark}
An (abstract) argumentation framework can be represented as a directed graph where nodes are arguments and an arrow from a node $A$ to a node $B$ represents an attack from argument $A$ against $B$.
\end{remark}

\begin{remark}
	Later in this dissertation assumptions that need to hold for a specific model are introduced. They are as well represented as a directed graph, but here an edge from assumption $A$ to model $B$ denotes that the assumption $A$ needs to hold so that $B$ is a possible model. However, the difference will be determinable from the context.
\end{remark}


\begin{exa}
 \autoref{fig:small_af} represents an \gls{AF} with the definition $AF = \langle \{A, B, C, D, E\}, \allowbreak \{(A,B), (B, C),\allowbreak (B, E), (C, B),\allowbreak (D,C), (D,D)\} \rangle $. This framework will be used as an example for the following definitions.
\end{exa}


\begin{figure}[!htb]
\centering
\begin{tikzpicture}[->,>=stealth',shorten >=1pt,auto,node distance=2cm,
                    thick]

  \node[main ] (A) {A};
  \node[main ] (B) [right of=A] {B};
  \node[main ] (C) [right of=B] {C};
  \node[main ] (D) [below of=C] {D};
  \node[main ] (E) [below of=B] {E};
  
  \path[every node/.style={font=\sffamily\small}]
    (A) edge node [left] {} (B)
    (B) edge [bend right] node [left] {} (C)
    	edge node [left] {} (E)
    (C) edge [bend right] node [left] {} (B)
    (D) edge node [left] {} (C)
        edge [loop right] node {} (D)
    ;
\end{tikzpicture}
\caption{Small example argumentation framework.}
\label{fig:small_af}
\end{figure}

\textbf{For the following definitions let $AF=\langle \S, \R \rangle, S' \subseteq \S$.}

\begin{definition}
	A subset $S'$ is \textbf{conflict-free} iff $ \forall A, B \in S': (A, B) \notin \R$ \textit{(the subset has no attacks between its arguments)}.
\end{definition}
\begin{remark}
	Conflict-free subsets are of interest, as these sets are not directly contradictory. In other words, a conflict-free subset of an argumentation framework does not contain any attacks between its members.
\end{remark}
\begin{exa}
Conflict-free subsets in \autoref{fig:small_af} are: $\emptyset, \{A\}, \{B\}, \{C, E\}, ...$. The sets $\{D\}, \{A, B\}, \{B, C\}, ...$ are not conflict-free.
\end{exa}

\begin{definition}
$A$ is \textbf{acceptable} \gls{w.r.t.} a subset $S'$ iff  $\forall B \in \S: (B, A) \in \R \Rightarrow \exists C \in S': (C, B) \in \R$ \textit{(an argument $A$ is acceptable with respect to a subset $S'$ iff for each attacker $C$ of $A$ there is an argument in $S'$ that attacks this attacker of $A$). }
\end{definition}

\begin{remark}
If an argument $A$ is acceptable \gls{w.r.t.} a subset $S'$, then there exists no holding counter-argument in this \gls{AF} causing the argument $A$ not to hold.
\end{remark}

\begin{lemma}
	$S'$ \textbf{defends} $X$, if and only if $X$ is acceptable with respect to $S'$.
\end{lemma}

\begin{exa}
In  \autoref{fig:small_af} $E$ is acceptable with respect to $\{A\}$. $A$ is acceptable with respect to $\emptyset$. 
\end{exa}


\begin{definition}
The \textbf{characteristic function} of an argumentation framework $AF$ (denoted as $F_{AF}$) is defined by the following:
	\begin{align*}
		&F_{AF}: 2^{\S} \rightarrow 2^\S &\\
		&F_{AF}(S') = \{A | A~\text{is acceptable with respect to}~S'\} &
	\end{align*}
\end{definition}
\begin{notation}
	If it is unambiguous, we often refer to $F_{AF}$ with $F$. 
\end{notation}

\begin{definition}
	A conflict-free set $S' \subseteq \S$ is \textbf{admissible} iff $S'$ defends all of its arguments (\textit{each argument in $S'$ is acceptable with respect to $S'$}) or $S' \subseteq F_{AF}(S')$.
\end{definition}

\begin{exa}
In the given example in \autoref{fig:small_af} sets $\{A\}, \{A, E\}$ are admissible. However $\{B\}$ is conflict-free but not admissible, since $B$ is not acceptable with respect to $\{B\}$.
\end{exa}

\begin{definition}
$S'$ is a \textbf{complete extension} iff  $S'$ is admissible and each argument that is acceptable with respect to $S'$ (which is defended by $S'$) belongs to $S'$.
\end{definition}

\begin{remark}
	A complete extension is a set of arguments that defends all members and includes all arguments that can be accepted regarding these members. In \autoref{fig:small_af} the set $S'=\{A\}$ is acceptable, but as this set defends as well $E$ (the only attacker $B$ is not acceptable \gls{w.r.t.} $S'$), this argument must be included in $S'$ to make the set complete. Hence $S' = \{A, E\}$ is a complete extension. 
\end{remark}


\begin{exa}
$X = \{A, E\}$ is a complete extension in \autoref{fig:small_af}, since it is admissible and it defends $A$ and $E$. Note that $C$ is not defended by $X$ since it does not attack $D$ and $D$ cannot be defended.
Complete extensions in \autoref{fig:pref_af} are $\{A\}, \{A, C\}$ and $\{A, D\}$.
\end{exa}


\begin{definition}
A grounded extension ($GE_{AF}$) of an argumentation framework $AF$ is the minimal (with respect to set inclusion) complete extension of $AF$. In other words $GE_{AF}$ is the least fixed point of $F_{AF}$ ($GE_{AF} = F_{AF}(\emptyset)$).
\end{definition}

\begin{remark}
	The grounded extension can be understood as the set of arguments an rational agent can \textit{accept without doubts}, as it contains only the minimal acceptable arguments for an argumentation framework and does not require the agent to assume anything about any argument.
\end{remark}

\begin{exa}
	The grounded extension of the example framework in \autoref{fig:small_af} is $\{A, E\}$.
	The grounded extension of the example framework in \autoref{fig:pref_af} is $\{A\}$.
\end{exa}


\begin{definition}
\label{def:preferred_extension}
A preferred extension of an argumentation framework $AF$ is a maximised (with respect to set inclusion) admissible set of $AF$.
\end{definition}

\begin{figure}[!htb]
\centering
\begin{tikzpicture}[->,>=stealth',shorten >=1pt,auto,node distance=2cm,
                    thick]

  \node[main] (A) {A};
  \node[main] (B) [right of=A] {B};
  \node[main] (C) [right of=B] {C};
  \node[main] (D) [right of=C] {D};
  \node[main] (E) [right of=D] {E};
  
  \path[every node/.style={font=\sffamily\small}]
    (A) edge node [left] {} (B)
    (C) edge node [left] {} (B)
    	edge [bend right] node [left] {} (D)
    (D) edge [bend right] node [left] {} (C)
	    edge [] node [left] {} (E)
    (E) edge [loop right] node {} (E)
    ;
\end{tikzpicture}
\caption{Argumentation framework used for some examples. }
\label{fig:pref_af}
\end{figure}

\begin{exa}
The preferred extensions in \autoref{fig:pref_af} are $\{A, C\}$ and $\{A, D\}$.
\end{exa}

\begin{definition}
A conflict-free set of arguments $S' \subseteq \S$ is called a \textbf{stable extension} iff $S'$ directly attacks each argument which does not belong to $S'$.
\end{definition}
\begin{exa}
The $AF$ in \autoref{fig:pref_af} has the stable extension $\{A, D\}$. $\{A, C\}$ is a preferred, but not a stable extension.
\end{exa}


\begin{remark}
The different extensions of a $AF = \langle \S, \R \rangle$ have the following relations between each other \cite{dung1995}:
\begin{itemize} 
	\item Each preferred extension is as well a complete extension.
	\item Each stable extension is as well a preferred extension.
	\item The grounded extension is the least (with respect to set inclusion) complete extension and therefore unique for each $AF$.
	\item Preferred extensions are the most (with respect to set inclusion) complete extensions.
	\item Arguments accepted in the grounded extension are skeptically accepted in the $AF$.
	\item Every $AF$ has at least one preferred extension.
	\item A stable extension does not always exist.
\end{itemize}
\end{remark}


\begin{definition}	
	An argument is regarded as \textbf{sceptically accepted under a semantic}, iff it is accepted in all extensions of this semantic (complete, grounded, preferred, stable). An argument is regarded as \textbf{credulously accepted under a semantic}, iff it accepted in at least one, but not all, extensions of this semantic. 	
\end{definition}

\begin{exa}
	In \autoref{fig:pref_af} $A$ is sceptically accepted under each semantic. $C, D$ are credulously accepted under a preferred semantic. $D$ is as well accepted sceptically under the stable semantic.
\end{exa}


Furthermore \cite{dung1995} introduces \textit{well-founded argumentation frameworks} (having exactly one extension which is grounded, preferred and stable), \textit{uncontroversial argumentation frameworks} and \textit{coherent argumentation frameworks} (each preferred extension of an $AF$ is stable). Regarding the problem we are looking at, defeasible argumentation is really important, as we are dealing with inconsistent \glspl{SKB}. Hence these restricted argumentation frameworks will not be used in this project, therefore they are not discussed any further.


In addition to the discussed extension-based semantics, \cite{liao} introduces a so called \textit{labelling-based approach} where there are usually three labels: \texttt{IN} (accepted argument), \texttt{OUT} (rejected argument) and \texttt{UNDEC} (undecided whether this argument is accepted or rejected) and a labelling function $\lambda: \S \rightarrow \{IN, OUT, UNDEC\}$.

By defining legally labelled arguments, definitions for \textit{conflict-free labelings},~\textit{admissible labelings} and \textit{complete/grounded/preferred labelings} can be derived. As there exists a bijective projection, they can be easily transferred to the extension-based semantics already introduced in this section. This labelling based approach has been used to implement the solving algorithm described in \cite{Modgil2009Proof, rodrigues} later in this thesis (see \autoref{sub:af_algorithm}).

