\section{Conclusion}
\label{sec:conclusion}
The goal of this project was to implement an already existing theoretical approach described in \cite{sassoon2016, sassoon2014, sassoon2016CD} by developing an intelligent agent, that is capable suggesting appropriate models to the end-user (usually a clinician). During the analysis of a particular data set different assumptions of these models must be evaluated to decide which are applicable. Furthermore, the application should present the relevant arguments for and against the various possible models  for a specific research question. To resolve the problem of defeasible and contradictory preferences between these models, different initialisations of multiple \glspl{EAF} based on \glspl{CD} have been used as described in \cite{sassoon2016, sassoon2016CD}.

The developed web-application provides an intuitive and easy-to-use approach to perform statistical model selection based on properties of the underlying data, statistical theory, as well as global and personal preferences. This allows statisticians with a deeper knowledge of statistical theory to provide their expertise to clinicians having in-depth insights about the acquisition and external properties of a data set. 

As a statistician has to enter the \gls{SKB} only once the workload for clinicians is reduced to a minimum. This allows them to focus on the actual analysis of the underlying data and equips them with a low-barrier tool, that they might be more likely to use over a longer period of time, which leads ultimately to a continuous data-driven decision process. As clinicians might not always be qualified to select the appropriated statistical models during the design stage of a study \cite{sassoon2014} a support-agent is essential for a data-driven process.

Only well known and acknowledged techniques have been used to evaluate the (extended) argumentation frameworks, which provides a well founded root for the overall decision process. In addition graphical representations of the used frameworks have been added to the web-application to provide easy access to the argumentation techniques, especially if they are not familiar to the end-user.

As described in \autoref{sub:outlooks}, the application could be improved by adding the functionality to provide data-provenance by providing "dynamic" \gls{R}-scripts (scripts that are binding to a mapping of database columns instead of names). 

As the underlying work of I. Sassoon \cite{sassoon2016CD} is still in progress, it is likely that further strategies regarding the selection of possible models and evaluating their preferences over one and another will be developed. As the application\footnote{Published under the \href{https://opensource.org/licenses/MIT}{MIT license}.} is written in the well-known framework \gls{RoR} and includes a sophisticated test suite, changes and extensions are easily possible.
 
Although the requirements for the project changed slightly (see \autoref{sub:preferences}) the agile software development process and the defined \glspl{use_case} allowed to react flexible on these changes and provided a good source for discussions with the supervisor and the scientific assistant of this project, which lead to no significant delays during the development process.