\section{Conclusion}
\label{sec:conclusion}
The aim of this project was to implement an already existing theoretical approach described in \cite{sassoon2014} by developing an intelligent agent, that is capable to suggest appropriate models to the end-user (usually a clinician). During the analysis of a particular dataset different assumptions that each of these models require must be evaluated and taken into account to decide which models are applicable. Furthermore the application should present the relevant arguments for and against the various possible models  for a specific research question. To resolve the problem of defeasible and contradictory preferences between these models, different initialisations of multiple \glspl{EAF} based on \glspl{CD} should be used as described in \cite{sassoon2016, sassoon2016CD}. The implementation of this intelligent agent should be done in \gls{RoR}.

The developed web-application provides an intuitive and easy-to-use approach to perform statistical model selection based on properties of the underlying data, statistical theory, as well as global and personal preferences. This allows statisticians with a deeper knowledge of statistical theory to provide their expertise to clinicians having in-depth insights about the acquisition and external properties of a dataset. 

As a statistician has to enter the \gls{SKB} only once the workload for clinicians is reduced to a minimum. This allows them to focus on the actual analysis of the underlying data and equips them with a low-barrier tool, that they might be more likely to use over a longer period of time which leads ultimately to a continuous data-driven decision process. Furthermore, clinicians might not be always qualified to select the appropriated statistical models during the design stage of a study \cite{sassoon2014}, thats why an support-agent is so essential for this data-driven process.

Only well known and acknowledged techniques have been used to evaluate the (extended) argumentation frameworks, which provides a well founded root for the overall decision process. In addition graphical representations of the used frameworks have been added to the web-application to provide easy access to the argumentation techniques, even though they might not be familiar to the end-user.

As described in \autoref{sub:outlooks}, the application could be drastically improved by adding the functionality to provide data-provenance by providing "dynamic" \gls{R}-scripts (scripts that are binding to a mapping of database columns instead of names). 

An additional functional extension might include the ability to fully document a data sets provenance and the context in which it was acquired. By implementing an additional anonymization functionality the agent could then as well be used as data-source for other projects related to the evaluation of clinical data sets and statistical models.

Furthermore, the developed solvers for \glspl{AF} and \glspl{EAF} have not been implemented in a generic way, extracting these into separate reusable \texttt{gems} could add benefits to other projects related to the theoretical approaches.

As the underlying work of I. Sassoon \cite{sassoon2016CD} is still in progress, it is likely that further strategies regarding the selection of possible models and evaluating there preferences over one and another will be developed. As the application\footnote{Published under the \href{https://opensource.org/licenses/MIT}{MIT license}.} is written in the well-known framework \gls{RoR} and includes a sophisticated test suite, changes and extensions are easily possible.
 
Although the requirements for the project changed slightly (see \autoref{sub:preferences}) the agile software development process and the defined \glspl{use_case} allowed to react flexible on these changes and provided an outstanding source for discussions with the supervisor and the scientific assistant of this project which lead to no significant delays during the development process.
\newpage
\newpage
 
 \tbd{
 \todo{Check the stylguide}
\
\\$\square${Captions must be provided for all figures and tables.}
\\$\square${Equations (or important equations), figures and tables must be numbered.}
\\$\square${All figures and tables must be referred to in the text.}
\\$\square${Units of all variables must be provided.}
\\$\square${Numercial values (floating-point number) should be in 4 decimal places.}
\\$\square${Contractions should not be used.}
\\$\square${Check punctuation of sentences.  In particular, those sentences with equation.  For example, if an equation is at the end of a sentence, a full stop should be used.}
\\$\square${All variables must be defined.}
\\$\square${Font face of variables throughout the report (in the text, equation, figures and table) must be consistent.}
\\$\square${Use proper headings for chapters, sections, subsections.}
\\$\square${Chapters, sections, subsections should be numbered and with the same numbering system throughout the report.}
\\$\square${It is suggested that vector and matrix variables should be in bold, scalar variables should be in italic.}
}