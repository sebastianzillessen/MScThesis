\subsection{Statistical Model Selection via Argumentation}

\label{sub:statistical_model_selection}

The goal of this project is to use argumentation theory for statistical model selection in mostly clinical environments. The demand for systems supporting clinicians during analysis of this data in their day-to-day practice is extending (because of increasing availability, growing size of data sets available for clinicians, and raising awareness on evidence based decision-making). In the following section a short summary on the underlying theory presented by Sassoon \textit{et al.} in \cite{sassoon2014} will be given, which will be extended by the solution to express the order of preferences in different \glspl{CD} \cite{sassoon2016,sassoon2016CD}.

During the design of clinical studies, clinicians often struggle with the selection of the right model to analyse the retrieved data as they might not be qualified to perform the statistical model selection required for their research question. The process of implementing models, specifying their requirements, providing the arguments for or against these models and specifying preferences between them should be separated from the actual design process and  done by a statistician. An intelligent model selection system, which is capable of suggesting appropriate model(s), would reduce administrative and demanding workload and empower clinicians to make data driven decisions.

Sassoon \textit{et al.} \cite{sassoon2014} address these problems with a defeasible knowledge base, as the (counter-) arguments for specific models might be contradicting. Furthermore the system, the clinician and the statistician might have different preferences over models that need to be expressed in this knowledge base. Therefore they propose to split the problem into two parts (i) a (defeasible) \textit{knowledge base} that contains the statistical model definitions, the objectives and assumptions of a model; (ii) \textit{argumentation schemes} to guide the model selection process and to represent expressed preferences, which are reflected by \gls{CD} specific preference orders proposed in later work \cite{sassoon2016,sassoon2016CD} .

The \textit{knowledge base} is used to instantiate the \textit{argumentation schemes} and defines how research objectives can be achieved through different statistical models considering their given assumptions. \textit{Research objectives} are defined as different 'families' of analysis (e.g. survival analysis or categorical outcome variable analysis).


\glsreset{SKB}
\subsubsection{Statistical Knowledge Base}
\label{sub:SKB}

The \gls{SKB} consists of objectives $\O=\{o_1, ..., o_u\}$ (different types of research questions), models $\M=\{m_1, ..., m_v\}$ and assumptions $\A = \{a_1, ..., a_w\}$. Models represent statistical analysis techniques employable to answer a research question. Assumptions are conditions that ought to be met to employ a model.

\begin{definition}
	Let $R_{\O\M}: \O \times \M$ be an m:n\footnote{Each objective can be achieved by one or more models, each model can answer one or more objectives.}-relationship such that $(o_i, m_j)\in R_{\O\M}$ implies objective $o_i$ can be achieved by means of model $m_j$. 
\end{definition}

\begin{definition}
	Let $R_{\M\A}: \M \times \A$ be the relation between models and their assumptions. $(m_i, a_j)\in R_{\M\A}$ implies  the model $m_i$ requires the assumption $a_j$ to be true to be applicable. Let $\A(m_i) = \{a_j | (m_i, a_j) \in R_{\M\A}\}$ be the set of assumptions of $m_i$.
\end{definition}

\begin{remark}
In contradiction to \cite{sassoon2014} we will not use the proposed approach to distinguish between \textit{critical} and \textit{non-critical} assumptions, as they have been used in the initial approach to express the preference over different possible models. Instead we gonna use the \gls{CD}-driven approach described in \cite{sassoon2016CD}.
\label{rem:cds}
\end{remark}


\begin{definition}
To apply a model $m_i$ all assumptions $\A(m_i) = \{a_j | (m_i, a_j) \in R_{\M\A} \}$ must be met\footnote{As described in \cref{rem:cds} we regard all assumptions as critical.}.
\end{definition}

Each \textit{assumption} will be either specified as a specific property of the data set (assessed by applying tests on the data set) or as a characteristic of the population of interest or the way in which the data set was collected from that population (relying on the expertise of a domain expert). Sassoon \textit{et al.} proposes a partitioning of all assumptions: $\A_t$ denotes the set of \textit{tests} (applying a test on the available data set). $\A_q$ denotes the set of \textit{queries} (assessed by asking the clinician for an opinion) \cite{sassoon2014}. Lets define $\A_t(m_i)= \{a_j| (m_i, a_j) \in \A_t\}$ and $\A_q(m_i)= \{a_j| (m_i, a_j) \in \A_q\}$. 

\autoref{fig:skb_example} shows the structure between \textit{objectives}, \textit{models} and \textit{assumptions} in a small example. The assumptions $\{a_1, a_2\} \in \A_t$ are based on the provided data set, $\{a_3, a_4\} \in \A_q$ are based on the domain expertise. Objectives $\{o_1, o_2\}$ can  be achieved by the possible model $m_1$, $o_3$ can only be achieved by $m_3$. 

\begin{figure}[tbp]
\centering
\begin{tikzpicture}[->,>=stealth',shorten >=1pt,auto,node distance=1cm,
                    thick]

  \node[main] (o1) {$o_1$};
  \node[main] (o2) [below of=o1] {$o_2$};
  \node[main] (o3) [below of=o2] {$o_3$};

  \node[main,n_fill_green] (m1) [right= 1.5cm of o1] {$m_1$};
  \node[main,n_fill_red] (m2) [below of=m1] {$m_2$};
  \node[main,n_fill_green] (m3) [below of=m2] {$m_3$};

  \node[main,n_fill_red] (a2) [right= 1.5cm of m1] {$a_2$};  
  \node[main,n_fill_green] (a1) [above of=a2] {$a_1$};
  \node[main,n_fill_green] (a3) [below of=a2] {$a_3$};
  \node[main,n_fill_green] (a4) [below of=a3] {$a_4$};
  
  \node (DB) [cylinder, shape border rotate=90,draw,minimum height=2cm,minimum width=1.3cm, fill=gray!10][right= 3cm of a2]{DB};
  \node (T) [draw=black,fill=gray!10,cloud,font=\fontfamily{ppl}\fontsize{1cm}{1.5cm}\selectfont][right= 3cm of a4]{?};
  
  \path[every node/.style={font=\sffamily\small}]
    (a1) edge [] node [left] {} (m1)
         edge [] node [left] {} (m2)
         edge [dashed] node [left] {} (DB)
    (a2) edge [] node [left] {} (m2)
	     edge [dashed] node [left] {} (DB)
    (a3) edge [] node [left] {} (m2)
    	 edge [dashed] node [left] {} (T)
    (a4) edge [] node [left] {} (m3)
    	 edge [] node [left] {} (m2)
    	 edge [dashed] node [left] {} (T)
    (m1) edge [] node [left] {} (o1)
         edge [] node [left] {} (o2)
	(m2) edge [] node [left] {} (o1)
	     edge [] node [left] {} (o2)
	(m3) edge [] node [left] {} (o3)
    ;
\end{tikzpicture}
\caption{Example of a \gls{SKB}. Green coloured assumptions hold while red coloured assumptions do not hold. Green coloured models are the possible models after \cref{as:1}.}
\label{fig:skb_example}
\end{figure}



\subsubsection{Adding Preferences for different Context Domains to the Statistical Knowledge Base}
\label{sub:preferences}

To achieve an objective $o_c$ (which has been selected by the clinician) a number of models $m_i$ might be possible providing all their assumptions $\A(m_i)$ are met. The process of instantiating the arguments can be seen in \cref{as:1}.

\begin{as}[tb]
	\centering
	\fbox{\begin{minipage}{0.7\textwidth}
	\begin{itemize}
		\item Model $m_i$ achieves objective $o_c$.
		\item The data set meets the set of assumptions $\A_t' = \A_t(m_i)$.
		\item The research project meets the set of assumptions $\A_q' = \A_q(m_i)$.
		\item $\A(m_i) \subseteq \A_t' \cup \A_q'$.
	\end{itemize}
	\rule{\textwidth}{0.5pt}\\
	$~~~~\Rightarrow$  $m_i$ is a possible model for the research question $o_c$.
	\end{minipage}}
	\caption{Constructed argument for a possible model \cite{sassoon2014}. 	\label{as:1}}

\end{as}

As quoted in \cref{rem:cds} we will not use the initial approach of instantiating AS2 \cite{sassoon2014}, but the \gls{CD}-based approach described in \cite{sassoon2016,sassoon2016CD}: An \gls{EAF} can be employed to capture and reason with statistical and research domain knowledge that affects the relative strength of arguments and thereby implies an order over different context domains. To do so, \glspl{CPAF} are used in combination with \glspl{EAF} by defining a preference ordering $Pref : \M \times \M$ for a given set of models $\M = \{m_1, ..., m_n\}$ by assigning performance measurements to each model for a given \gls{CD}. A sample performance function for model resilience to censoring (CD1) can be seen in \autoref{tab:cd1}. An additional performance function based on the intention of the model (CD2) can be seen in \autoref{tab:cd2} (both taken from \cite{sassoon2016CD}).


These orders over the models (see \autoref{tab:cd1} and \ref{tab:cd2}) can then be transferred into preference arguments (see \autoref{tab:orders}). 

\begin{table}[htbp]
	\centering
	\begin{tabular}{|l|c|l|c|}
	\hline
	Context Domain 			& Model 			& Performance measure & Order\\
	\hline\hline
	\multirow{3}{*}{absent} & $m_1$ KM 		& unaffected		& 	\multirow{3}{*}{$m_1, m_2, m_3$}\\
							& $m_2$ PH		& unaffected		&\\
							& $m_3$ $X^2$	& unaffected		&\\
	\hline
	\multirow{3}{*}{light} 	& $m_1$ KM 		& unaffected		& \multirow{3}{*}{$m_3 \prec m_1, m_2$}\\
							& $m_2$ PH		& unaffected		&\\
							& $m_3$ $X^2$	& affected		&\\
	\hline
	\multirow{3}{*}{heavy} 	& $m_1$ KM 		& affected		& \multirow{3}{*}{$m_1, m_3 \prec m_2$}\\
							& $m_2$ PH		& unaffected		&\\
							& $m_3$ $X^2$	& affected		&\\
	\hline
	\end{tabular}
	\caption{Sample performance function for model resilience to censoring (CD1).}
	\label{tab:cd1}
\end{table}


\begin{table}[htbp]
	\centering
	\begin{tabular}{|l|c|l|c|}
	\hline
	Context Domain 			& Model 			& Performance measure & Order\\
	\hline\hline
	\multirow{3}{*}{predict}& $m_1$ KM 		& avoid		& \multirow{3}{*}{$m_1, m_3 \prec m_2$}\\
							& $m_2$ PH		& suitable	&\\
							& $m_3$ $X^2$	& avoid		&\\
	\hline
	\multirow{3}{*}{explain}& $m_1$ KM 		& suitable	& \multirow{3}{*}{$m_1, m_2 \prec m_3$}\\
							& $m_2$ PH		& suitable	&\\
							& $m_3$ $X^2$	& neutral	&\\
	\hline
	\end{tabular}
	\caption{Sample performance function for model intent (CD2).}
	\label{tab:cd2}
\end{table}

\begin{table}[hbtp]
	\begin{align*}
		CD1_{light} &\twoheadrightarrow (m_3 \rightarrow m_1) & CD2_{predict} &\twoheadrightarrow (m_1 \rightarrow m_2)\\
		CD1_{light} &\twoheadrightarrow (m_3 \rightarrow m_2) & CD2_{predict} &\twoheadrightarrow (m_3 \rightarrow m_2)\\
		CD1_{heavy} &\twoheadrightarrow (m_1 \rightarrow m_2) & CD2_{explain} &\twoheadrightarrow (m_1 \rightarrow m_3)\\
		CD1_{heavy} &\twoheadrightarrow (m_3 \rightarrow m_2) & CD2_{explain} &\twoheadrightarrow (m_2 \rightarrow m_3)\\
	\end{align*}
	\caption{Orders over models derived from the two different \glspl{CD} (see \Cref{tab:cd1,tab:cd2}).}
	\label{tab:orders}
\end{table}

\begin{figure}[hbtp]
	\centering
	\begin{tikzpicture}[->,>=stealth',shorten >=1pt,auto,node distance=3cm,
                    thick]
  	\node[n_fill_red] (m1) {$m_1$};
	\node[n_fill_green] (m2) [below left of=m1]{$m_2$};
	\node[n_fill_red] (m3) [below right of=m1]{$m_3$};

  \path[every node/.style={font=\sffamily\small}]
    (m1) edge [bend right,dashed] coordinate [pos=0.5] (m1m2) node [left] {} (m2)
    (m1) edge [] coordinate [pos=0.5] (m1m3) node [left] {} (m3)
    (m2) edge [] coordinate [pos=0.5] (m2m1) node [left] {} (m1)
    (m2) edge [bend right] coordinate [pos=0.5] (m2m3) node [left] {} (m3)
    (m3) edge [bend right] coordinate [pos=0.5] (m3m1) node [left] {} (m1)
    (m3) edge [dashed] coordinate [pos=0.5] (m3m2) node [left] {} (m2)
    ;
	\node[box] (CD1heavy1)[above left of=m1m2,xshift=1cm,yshift=-1cm] {$CD1_{heavy}$};
	\node[box] (CD1heavy2)[below of=m3m2, yshift=1cm] {$CD1_{heavy}$};  
    
   \path [every node/.style={red}] 
   (CD1heavy1) edge [->>] node [left] {} (m1m2)
   (CD1heavy2) edge [->>] node [left] {} (m3m2)
   ;
\end{tikzpicture}

	\caption{\gls{EAF} if $CD1_{heavy}$ holds: The attacks $(m_1,m_2)$ and $(m_3,m_2)$ are cancelled out and $m_3$ is the only acceptable model w.r.t. to $\S_1' = \{CD1_{heavy}\}$.}
	\label{fig:cd1_heavy}
\end{figure}

Sassoon \textit{et al.} propose further to assign each \gls{CD} a priority, so that they get evaluated one after the other, until one final (if at all possible) preferred model remains. This order should be based on the global relevance of the \glspl{CD}: Statistical Theory $\succ$ Intention of Analysis $\succ$ Clinicians Preference. 

After evaluating the possible models by using \autoref{as:1} we will generate an \gls{AF} where each of the possible model attacks every other model. Then each \gls{CD} will be initiated and adds its attacks on the existing \gls{AF} and translates it into an \gls{EAF}. This can then be evaluated w.r.t. the argument holding in this \gls{CD} (e.g. $\S_1' = \{CD1_{heavy}\}$, see \cref{fig:cd1_heavy}). If a final decision (only one model is acceptable w.r.t. $\S_1'$) can be made, the process terminates. 


\begin{figure}[hbtp]
	\subfigure[0.45\textwidth][First \gls{CD} applied: $CD1_{light}$ holds. The attacks $(m_3,m_2)$ and $(m_3,m_1)$ are cancelled out. $m_3$ is not acceptable w.r.t. $\S_2' = \{CD1_{light}\}$.]{
	  	\begin{minipage}{0.45\textwidth}
		  	\centering
		  	\begin{tikzpicture}[->,>=stealth',shorten >=1pt,auto,node distance=3cm,
                    thick]
  	\node[n_fill_green] (m1) {$m_1$};
	\node[n_fill_green] (m2) [below left of=m1]{$m_2$};
	\node[n_fill_red] (m3) [below right of=m1]{$m_3$};

  \path[every node/.style={font=\sffamily\small}]
    (m1) edge [bend right] coordinate [pos=0.5] (m1m2) node [left] {} (m2)
    (m1) edge [] coordinate [pos=0.5] (m1m3) node [left] {} (m3)
    (m2) edge [] coordinate [pos=0.5] (m2m1) node [left] {} (m1)
    (m2) edge [bend right] coordinate [pos=0.5] (m2m3) node [left] {} (m3)
    (m3) edge [dashed,bend right] coordinate [pos=0.5] (m3m1) node [left] {} (m1)
    (m3) edge [dashed] coordinate [pos=0.5] (m3m2) node [left] {} (m2)
    ;
	\node[box] (CD1light1)[above right of=m3m1,xshift=-1cm,yshift=-1cm] {$CD1_{light}$};
    
   \path [every node/.style={red}] 
   (CD1light1) edge [->>] node [] {} (m3m1)
   (CD1light1) edge [bend right,->>] node [] {} (m3m2)
   ;
\end{tikzpicture}

		  	\end{minipage}
		\label{fig:cd1_light}
	}
	\hfill
	\subfigure[0.45\textwidth][Next \gls{CD} applied: $CD2_{predict}$ holds. The attack $(m_1,m_2)$ is cancelled out and $m_1$ is not possible anymore. Hence $m_2$ is the recommended model.]{
	  	\begin{minipage}{0.45\textwidth}
	  		\centering
			\begin{tikzpicture}[->,>=stealth',shorten >=1pt,auto,node distance=3cm,
                    thick]
  	\node[n_fill_red] (m1) {$m_1$};
	\node[n_fill_green] (m2) [below left of=m1]{$m_2$};
  \path[every node/.style={font=\sffamily\small}]
    (m1) edge [bend right, dashed] coordinate [pos=0.5] (m1m2) node [left] {} (m2)
    (m2) edge [bend right] coordinate [pos=0.5] (m2m1) node [left] {} (m1)
    ;
	\node[box] (CD2predict)[left of=m1m2,xshift=1cm] {$CD2_{predict}$};
	
   \path [every node/.style={red}] 
   (CD2predict) edge [->>] node [left] {} (m1m2)
   ;
\end{tikzpicture}
			\\
		  	\hspace*{70pt}
		\end{minipage}
		\label{fig:cd2_predict}
	}
	\caption{\gls{EAF} generated for three possible models with \glspl{CD} $CD1_{light}$ and $CD2_{predict}$.}
	\label{fig:cd1_light_cd2_predict}
\end{figure}



\autoref{fig:cd1_light_cd2_predict} shows another example, where we need to evaluated multiple \glspl{CD} to get a final result. First the \gls{EAF} will be evaluated w.r.t. the argument holding in the first \gls{CD} $\S_1' = \{CD1_{light}\}$ as shown in \Cref{fig:cd1_light}, then the second \gls{CD} $\S_2'' = \{CD2_{predict}\}$ is applied in \Cref{fig:cd2_predict}. This is continued until no further preferences are defined or only one -- the preferred -- model remains.

During this process the \glspl{CD} that relate to statistical theory are regarded as more important than clinician preferences. This is reflected in the final web application by reserving the priorities 1 - 9 to the statistician only (1 has the highest priority).

