\section{Introduction}

Nowadays data collection is omnipresent and the buzzword \textit{big data}\footnote{Big data is often described by the five V's: Volume, Variety, Velocity, Variability, Veracity \cite{Hilbert2015}} is referred to be the next "organizational challenge" most of the industry is or will be facing \cite{bigdata}. However, most of the research is done on the extraction of information from large data sets (so called Big Data Analysis). Therefore small data sets collected in day-to-day practice of professionals are often overlooked. 

Clinicians and hospitals collect a lot of data on their patients, the used therapies and the outcomes. Unfortunately, this data is often not used to improve future practice. A recent systematic review has shown, that the usage of statistical analysis has improved the survival analysis slowly \cite{survivalAnalysis}. However, this increased usage of statistical models in theory (especially survival analysis) did not result in any noticeable evidence of assumption testing prior to the use of a model in actual clinical studies.

This project aims to implement an intelligent agent that provides advice based on statistical theory on the analysis of such data. The system depends on the design described in the related papers by Sassoon \textit{et al.} \cite{sassoon2016,sassoon2014,sassoon2016CD}. In particular, the most recent publication will be used as solution to the problem of preferences over models in different context domains. 


This thesis will first clarify the project aims and objectives and the used methodologies in subsections \ref{sub:aims} and \ref{sub:methodologies}. This is followed by a background research in \autoref{sec:background} providing a review of the theoretical aspects of argumentation frameworks, their extensions and the theory underlying statistical model selection. 
\Cref{sec:projectplan} presents a detailed project plan and the following \autoref{sec:design} introduces the reader to the used design process. Furthermore, \autoref{sec:specification} lists the specifications and a detailed list of requirements of the project. The developed application itself will be introduced to the reader in \autoref{sec:app}. The consecutive chapter (see \autoref{sec:implementation}) focuses on the actual implementation of the software as an \gls{RoR} web application. 
The report is concluded by a critical evaluation (see \autoref{sec:evaluation}) of the project and the delivered web application and an overall conclusion (see \autoref{sec:conclusion}).
\Cref{app:a,app:b,app:c,app:d} contain further material related to this project.


\subsection{Project Goals and Objectives} 
\label{sub:aims}

The goals of this project can be divided into a list of primary and secondary. For a successful project progression the following primary objectives have to be reached:
\begin{itemize}
	\item General explanation and summary of \glspl{AF}, \glspl{EAF}, statistical model selection and the definition of preferences between models related to \glspl{CD}.
	\item Development of an \gls{RoR} web application that implements the requirements proposed in \cite{sassoon2014,sassoon2016CD} including but not limited to:
	\begin{itemize}
		\item An approach to instantiate and solve \glspl{AF} and \glspl{EAF}.
		\item The ability to store, manage and reuse research questions, analysis and preferences for statistical models on different data sets.
		\item An easy to use user interface to upload data collected during clinical studies and run analyses in an interactive way.
		\item The ability to deal with preferences between models on a meta-level using \glspl{EAF} while taking into account global and personal (end user) preferences. The approach proposed in \cite{sassoon2016CD} involving \glspl{CD} will be used.
		\item A user rights management to allow the system to be used by clinicians, statisticians and super-users (admins).
		\item A small set of statistical models and their assumptions integrated in the system (provided by Sassoon in \cite{sassoon2016CD}).
		\item A comprehensive set of unit and integration tests of the system.
		\item Hosting of this web application at a public accessible provider.
	\end{itemize}	
	\item The system should provide the end user with an explanation why a statistical model should be used, and why one model might be preferred over another one.
\end{itemize}

\bigskip

The secondary goals are desired to be achieved but do not influence the successful finalisation of the project. These objectives are the following:

\begin{itemize}
	\item A documentation of the developed system providing information on how to use it and an overview over the key components of the application.
	\item A reusable implementation to solve standard \glspl{AF} in Ruby as a \texttt{gem} including documentation and a comprehensive set of unit tests.
	\item A reusable implementation to solve \glspl{EAF} in Ruby as a \texttt{gem} including documentation and a comprehensive set of unit tests.
	\item Extended sets of statistical models and their assumptions.
	\item A graphical representation of the arguments explaining the actual analysis outcome of the system.
\end{itemize}


\subsection{Methodologies of the Project}
\label{sub:methodologies}
This project will be developed in an agile way. To ensure that it meets the requirements described by Sassoon \textit{et al.} (\cite{sassoon2016,sassoon2014, sassoon2016CD}, see \autoref{sub:statistical_model_selection}), the main author of those papers is treated as a client or \gls{product owner} during the requirements analysis and the testing phase. For the actual development process the Use-Case 2.0 approach by Jacobson \textit{et al.} \cite{jacobson2011usecase} is used as it provides a great way to communicate, specify and iterate over functional (independent) parts of the system with non-developers. Due to its descriptive nature it does not require any knowledge about the actual process to be easily understandable. However, this methodology will be explained and introduced in detail later in this thesis (see \autoref{sec:design}). 

As a project communication and management tool Trello\footnote{\url{http://www.trello.com}} is utilised as it provides an easy-to-use and interactive way of dealing with cards (in our case \glspl{use_case} and tasks) and to group them. During the project planning it was decided that the development would be broken down into four release cycles (RC~1 - 4, see \autoref{sec:projectplan}), as this will provide a modularisation of the project and allows early feedback on it. However, tracking of the already achieved intermediate steps and the actual progress of the development process can be done efficiently with Trello. Labels and different lists visualise the status and progress of each task and \gls{use_case} (see \autoref{fig:trello}).


\begin{figure}[h]
\centering
	\includegraphics[page=1,width=\textwidth]{figures/trello}
\caption{Screenshot of the used Trello board used as project management tool.}
\label{fig:trello}
\end{figure}

\subsection{Supplementary Resources}

During this Masters Project a web-application was developed, which is publicly available on \href{http://small-data-analyst.herokuapp.com}{http://small-data-analyst.herokuapp.com}. Users can sign-up (approval of an administrator required, please reach out to the author if you have any questions related to that) and upload their own data sets. Some of the existing research questions are shared between all users of the application. However -- if required -- the source code is available on \href{https://github.com/sebastianzillessen/small-data-analyst}{Github}\footnote{\url{https://github.com/sebastianzillessen/small-data-analyst}} and an installation instruction can be found in the appendix \autoref{app:installation}. A detailed list of used third-party applications is attached in \autoref{app:3rdparty}.

