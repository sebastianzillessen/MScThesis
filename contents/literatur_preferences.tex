\subsection{Argumentation Theory - Working with Preferences in $AF$}

A Dung's argumentation framework is based on logical theory which is transformed to arguments and a binary attack relation. By applying the different extensions on the created argumentation framework accepted arguments can be evaluated. This approach does not take in account, that for practical reasons and in some applications only one unique set of accepted arguments has to be determined, taking into account that because of personal preferences, contextual requirements or additional information one argument might have a higher priority and is preferred over another. This order of preferences is often itself defeasible and conflicting and therefor subject to argumentation \cite{Modgil2009}.

\newacronym{PAF}{PAF}{Preference-based argumentation framework}
\newacronym{VAF}{VAF}{Value-based argumentation framework}
\newacronym{EAF}{EAF}{Extended argumentation framework}

\begin{figure}[h]
\centering
\begin{tikzpicture}[->,>=stealth',shorten >=1pt,auto,node distance=2cm,
                    thick,main node/.style={circle,draw,font=\sffamily\Large\bfseries}]

  \node[main node] (A) {A};
  \node[main node] (B) [left of=A] {B};
  \node[main node] (C) [above right of=A] {C};
  \node[main node] (D) [below right of=A] {D};
  \node[main node] (E) [below right =0.8cm and 1.6cm of C] {E};
  
  \path[every node/.style={font=\sffamily\small}]
    (A) edge [bend right, dashed] coordinate [pos=0.5] (AB) node [left] {} (B)
    (B) edge [bend right] coordinate [pos=0.5] (BA) node [left] {} (A)
    (C) edge [bend right] coordinate [pos=0.5] (CD) node [left] {} (D)
    (D) edge [bend right, dashed] coordinate [pos=0.5] (DC) node [left] {} (C)
    ;
   \path [every node/.style={red}] 
   (C) edge [->>, bend right] node [left] {} (AB)
   (D) edge [->>, bend left, dashed] node [left] {} (BA)
   (E) edge [->>] node [left] {} (DC)
   ;
\end{tikzpicture}
\caption{\gls{EAF} with preferences over arguments. Dashed attacks are canceled out. Double-arrow-headed edges represent attacks on attacks.}
\label{fig:eaf_intro}
\end{figure}

Other approaches to deal with preferences in argumentation frameworks have been proposed by \cite{amgoud,amgoud1998,Bench2003}. A \gls{PAF} defined in \cite{amgoud1998} is  a triple $\langle A, R, Pref \rangle$ where $Pref$ is a (partial of complete) order preordering on $A \times A$. The difference between this approach and the one we use is mostly the requirement of a strict ordering that has to be associated with $Pref$ and must be explicitly defined.

\glspl{VAF} as proposed in \cite{Bench2003} define an argumentation framework as a 5-tuple: $\langle S, R, V, val, P \rangle$ ($S$: Arguments, $R$: Attack relation, $V$: nonempty set of values, $val(\cdot): S \rightarrow V$: value mapping function, $P$: possible set of audiences). The set of audiences $P$ is introduced to be able to make use of preferences between values in $V$, so we might have as many audiences as there are orderings on $V$. The new definition of an argument that defeats another argument takes in account the audience $a$ and the $val(\cdot)$ of both arguments to define successful attacks. This approach requires an in our case unknown value mapping function $val(\cdot)$ and doesn't argue with preferences in a natural way.

However, we decided to go with the \gls{EAF} approach introduced in \cite{Modgil2009}, as it provides a useful meta-level on preferences between other arguments, by extending the basic framework with a new attack relation between arguments and attacks. This overcomes the issues of defining orders of preferences or value-evaluation functions and enables us to argue about preferences regardless wether they are defeasible or conflicting. \autoref{fig:eaf_intro} shows an example for a \gls{EAF} representing the following arguments (taken from \cite{Modgil2009}):
\begin{itemize}
	\item $A$: "Today will be dry in London since the BBC forecast sunshine"
	\item $B$: "Today will be wet in London since CNN forecast rain"
	\item $C$: "But I think the BBC are more trustworthy than CNN"
	\item $D$: "However, statistically CNN are more accurate forecasters than the BBC"
	\item $E$: "Basing on a comparison on statistics is more rigorous and rational than basing a comparison on your instincts about their relative trustworthiness"
\end{itemize}

The default argumentation framework by \cite{dung1995} has been extended in \cite{Modgil2009} with a meta-level argumentation layer that claims preferences between other arguments. In the following section a short introduction to this framework extension will be given. 

\glsreset{EAF}
\begin{definition}
	An \gls{EAF} is a triple $\langle \S, \R, \D \rangle$ with $\S$ being a set of arguments and:
	\begin{itemize}
		\item $\R \subseteq \S \times \S$: attack relation.
		\item $\D \subseteq \S \times \R$: new attack relation on attacks.
		\item $\{(A, (B, C)), (A', (C, B))\} \subseteq \D \rightarrow \{(A, A'), (A', A)\} \subseteq \R$ (any arguments expressing contradictory preferences must attack each other).
	\end{itemize}
\end{definition}

\begin{remark}
"If $A$ attacks $(B, C) \in \R$ then $A$ expresses, that $C$ is preferred to $B$. If $A'$ attacks $(C, B)$, then $A'$ expresses that $B$ is preferred to $C$. Hence \glspl{EAF} are required to conform to the constraint that any such arguments expressing contradictory preferences must attack each other."\cite{Modgil2009}.
\end{remark}

Let $\Delta = \langle \S, \R, \D \rangle$ be a \gls{EAF} and $S' \subseteq \S$ for the following definitions.

\begin{definition}
$A$ \textbf{defeats$_{S'}$} $B$ iff $(A, B) \in \R$ and $\not \exists C \in S': (C, (A, B)) \in \D$. If $A$ defeats\low{S'} $B$ and $B$ does not defeat\low{S'} $A$ then $A$ \textbf{strictly} defeats\low{S'} $B$.
\end{definition}

\begin{notation}
For the rest of the document $A \rightarrow^{S'} B$ denotes that $A$ defeats\low{S'} B and $A \nrightarrow^{S'} B$ denotes that $A$ does not defeat\low{S'} $B$.
\end{notation}


By using this definition, similar properties as in Dung's argumentation framework can be introduced and defined.

\begin{definition}
	$S'$ is \textbf{conflict free} iff $\forall A, B \in S': (A, B) \in \R \Rightarrow (B, A) \notin \R \wedge \exists C \in S': (C, (A, B)) \in \D$ (a subset is only conflict free, if for every attack within the subset there is no counter attack and the attack itself is canceled out by an attack on this attack from an argument that is part of the subset as well).
\end{definition}


\begin{figure}[h]
\centering
\begin{tikzpicture}[->,>=stealth',shorten >=1pt,auto,node distance=2cm,
                    thick,main node/.style={circle,draw,font=\sffamily\Large\bfseries}]

  \node[main node] (C) [above right of=A] {C};
  \node[main node] (A) [below left of=C]{A};
  \node[main node] (B) [below right of=C] {B};
  
  \path[every node/.style={font=\sffamily\small}]
    (A) edge [] coordinate [pos=0.5] (AB) node [left] {} (B)
    ;
   \path [every node/.style={red}] 
   (C) edge [->>] node [left] {} (AB)
   ;
\end{tikzpicture}
\caption{\gls{EAF} with $\langle \S, \R, \D \rangle = \langle \{A, B, C\}, \{(A, B)\}, \{(C, (A, B))\} \rangle$.}
\label{fig:eaf_small}
\end{figure}

\begin{exa}
	The set $S' = \{A, B\}$ of the \gls{EAF} in \autoref{fig:eaf_small} is not conflict-free. But the set $S' = \{A, B, C\}$ is conflict-free as $C$ attacks the attack between $A$ and $B$ and cancels it out.
\end{exa}

\begin{lemma}
Let $S'$ be a conflict-free subset of $\S$ in $\langle \S, \R, \D \rangle$. Then for any $A, B \in S'$ $A$ does not defeat\low{S'} $B$.
\end{lemma}

\begin{definition}
$R_{S'} = \{X_1 \rightarrow^{S'} Y_1, ..., 	X_n \rightarrow^{S'} Y_n\}$ is called a \textbf{reinstatement set} for $C \rightarrow^{S'} B$ ($C$ defeats\low{S'} $B$), iff:
\begin{itemize}
	\item $C \rightarrow^{S'} B \in R_{S'}$,
	\item for i=1 ... n, $X_i \in S'$,
	\item $\forall X \rightarrow^{S'} Y \in R_{S'}, \forall Y' s.t. (Y', (X, Y))\in \D$, there is a $X' \rightarrow^{S'} Y' \in R_{S'}$.
\end{itemize}
\end{definition}


