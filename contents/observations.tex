\subsection{Observation and Feedback}
\sloppy

The analyses on the provided datasets (as described in \autoref{app:dataset}) are executed rather fast on even larger datasets ($400 \times 15$), as the actual data analysis is done in  \gls{R} scripts that have been used in statistical environments for many years and proofed to be efficient. The results for these scripts are pre-calculated in background processes as soon as the datasets are available. The actual analyses of the different research questions can therefore be performed rather quickly and will be stored for future reference in the database.

The final version of the developed application has been as well tested and evaluated by Isabel Sassoon, scientific assistant of this thesis. As a statistician with a practical knowledge in data mining and consulting clinicians, she could deliver the following feedback on the application (personal communication, August 2016):

In general the developed app fulfils the original defined requirements. In some aspects it provides even more functionality, such as the implementation of contextual factors and argumentation graphs. The overall layout is the same for the expert and the novice user with some of the advanced options being not available for the novice user profile.

Research question can be private or shared, which supports clinicians that are analysing the same data to compare and discuss the best approaches. However, this functionality could be extended in the future to provide the ability to publish research questions to groups of collaborators.


\subsection{Outlooks}
\label{sub:outlooks}

The developed application should be considered as a functional proof of concept and could be extended and improved in various ways. In the following subsection a short list of outlooks is provided (partly based on the Feedback of Isabel Sassoon, personal communication, August 2016).

For a prospective trial roll out to clinicians a pre-population of the application with data (including  all the relevant research questions, models, assumptions and preferences) would be required, as this will provide deeper feedback on the app from an end user perspective.

In addition to this roll out, a trial with statisticians would provide feedback on the underlying mechanisms of the process. This would as well give the chance to verify the system as a statistician could assess whether the recommended models the system reports are the same as he would recommend. 

The results of both above described roll outs could provide a comprehensive analysis of the app and the used methodology. The comparison against a range of criteria could then help to shape any future development plans.

Moreover, an output functionality which covers the steps taken from the research question selected, to the data used, the assumptions tested and the results of the assumptions testing as a summary document could provide better explanation of the process and could be used to develop a statistical analysis plan document (SAP).

Furthermore, as a very demanding client customisation of the layout and the phrasing of texts might be desired to reflect individual workflows in a more efficient way. This could be done during the client specific analysis and be part of a detailed design document.


Especially the process of adding new assumptions and models to research questions can be further improved. For now assumptions rely heavily on the underlying structure of the dataset. To perform two analysis regarding the research questions \textit{"Is there a difference between the survival time for a treatment group in accordance with their {\tiny $\begin{cases} age\\weight \end{cases}$} "} we would need to add the related test assumptions twice into the system. 

This is due to the fact, that the column names in the original datasets should not be changed without documentation as this would conflict the properties of data-provenance \cite{provenance}. As the only changing aspect between the two research questions are the column names \texttt{age} and \texttt{weight}, it would be simpler to map a column name of the data set to a specific column that is required by an assumption which would reduce the number of required assumptions in the system by preventing duplication. For an analysis, the clinician would then have to chose which columns of the dataset should be mapped to which data attribute required by the underlying assumptions. This feature would add high benefits regarding flexibility into the already existing application.
