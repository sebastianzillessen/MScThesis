\subsection{Observation and Feedback}
\sloppy

The analyses of the provided data sets (as described in \autoref{app:dataset}) are executed with good performance on even larger data sets ($\approx$ 400 rows $\times$ 15 columns). The actual data analysis is done in \gls{R}, which has been used in statistical environments for many years and proofed to be efficient. The results for these scripts are pre-calculated in background processes as soon as the data sets are available. Hence, the actual analyses of the different research questions can be performed quickly and is stored for future reference.

The final version of the developed application was tested and evaluated by Isabel Sassoon, scientific assistant of this thesis and product owner during the development process. As a statistician with practical knowledge in data mining and consulting clinicians, she could deliver the following feedback on the application (personal communication, August 2016):

In general the developed app fulfils the original defined requirements. In some aspects it provides even more functionality, such as the implementation of contextual factors and argumentation graphs. The overall layout is the same for the expert and the novice user with some of the advanced options being not available for the novice user profile.

Research questions can be private or shared, which supports clinicians that are analysing the same data to compare and discuss the best approaches. However, this functionality could be extended in the future to provide the ability to publish research questions to groups of collaborators.


\subsection{Outlooks}
\label{sub:outlooks}

The developed application should be considered as a functional proof of concept and could be extended and improved in various ways. A short list of outlooks is partial based on the feedback of Isabel Sassoon (personal communication, August 2016) follows:

For a prospective trial roll out to clinicians a pre-population of the application with data (including  all the relevant research questions, models, assumptions and preferences) would be required, as this will provide deeper feedback on the app from an end user perspective. In addition to this roll out, a trial with statisticians would provide feedback on the underlying mechanisms of the process. This would as well give the chance to verify the system as a statistician could assess whether the models the system reports are the same as he would recommend. The results of both roll outs could provide a comprehensive analysis of the app and the used methodology. The comparison against a range of criteria could then help to shape any future development plans.

Moreover, an output functionality that covers the steps taken from the research question selected, to the data used, the assumptions tested and the results of the assumptions testing as a summary document could provide better explanation of the process and be used to develop a statistical analysis plan document (SAP).

Furthermore, as a very demanding client, customisation of the layout and the phrasing of texts might be desired to reflect individual workflows. This could be done during the client specific analysis and be part of a detailed design document.


Especially the process of adding new assumptions and models to research questions can be further improved. For now assumptions rely heavily on the underlying structure of the data set. To perform two analysis regarding the research questions \textit{"Is there a difference between the survival time for a treatment group in accordance with their \textbf{age} / \textbf{weight}"} we would need to add the related test assumptions twice into the system. 

This is due to the fact, that the column names in the original data sets should not be changed without documentation as this would conflict the properties of data-provenance \cite{provenance}. As the only changing aspect between the two research questions are the column names \texttt{age} and \texttt{weight}, it would be simpler to map a column name of the data set to a specific variable name. This variable could then be used by an assumption during the evaluation of its associated \gls{R}-script. As a result, the number of required assumptions in the system would reduce drastically by preventing duplication. For each analysis, the clinician would then have to chose, which columns of the data set should be mapped to which variable required by the underlying assumptions. This feature would improve the developed application by adding flexibility and data-provenance to the system.

An additional functional extension might include the ability to fully document a data set provenance and the context in which it was acquired. By implementing an additional anonymisation functionality the agent could then as well be used as data-source for other projects related to the evaluation of clinical data sets and statistical models.

Furthermore, the developed solvers for \glspl{AF} and \glspl{EAF} have not been implemented in a generic way. Extracting these into separate reusable \texttt{gems} could add benefits to other projects related to the theoretical approaches.
