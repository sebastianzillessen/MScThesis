\section{Background Research}
\label{sec:background}

\todo{These four paragraphs are confusing. I suggest you first explain the problem and process of statistical model selection, without offering a solution. then talk about which techniques are applied and then in 2,1 you provide an overview of the techniques.}

The following section provides an introduction on \glspl{AF} and \glspl{EAF}. The approach for a statistical model selection proposed in \cite{sassoon2014} is based on computational models of argumentation. Usually multiple assumptions have to be fulfilled for statistical models to be applicable on a dataset. These assumptions are defeasible and may lead to multiple possible models, which requires argumentation over assumptions and models to provide a reasonable statistical model selection. An introduction to \glspl{AF} is given in \autoref{sub:dung}. 

Non-monotonic argumentation (as proposed in \cite{liao,dung1995}) and monotonic (classic) logic (as proposed in \cite{Reiter1980}) are in general different approaches to deal with reasoning. However, recent research is focusing on dialogue based approaches which are mostly based on non-monotonic argumentation \cite{parsons2000,Walton1995}. 
Nevertheless Sassoon \textit{et al.} propose to employ preferences by using \glspl{EAF} to reason about the order of applicable models and to deliver a final statistical model which should be used. 

To understand this approach, an extension to the standard framework to reason over preferences is presented later in this chapter (see \autoref{sub:eaf}). Other possible solutions to argue over preferences are explained in \autoref{sub:paf}.

Finally, a summary of Sassoon \textit{et al.} \cite{sassoon2014} on the problem of finding applicable and choosing preferred models by clinicians for a given research question is provided in \autoref{sub:statistical_model_selection}. In our application we focused on the approach described by the most recent papers \cite{sassoon2016,sassoon2016CD} which will be summarised as well.


\subsection{Argumentation Theory: General introduction}
\label{sub:dung}
In the following section a general overview on \glspl{AF} will be given. The notation and definitions are based on Dung's theory \cite{dung1995} as it is a widely used definition for argumentation frameworks and the main sources of this dissertation \cite{sassoon2014,sassoon2016CD,Modgil2009} are based on this approach.
\begin{definition}
	An argumentation framework is a tuple $AF = \langle \S, \R \rangle$ where $\S$ is a set of arguments and $ \R \subseteq \S \times \S$. $\R$ is a binary relation and called attack relation.
\end{definition}


An (abstract) argumentation framework can be represented as a directed graph where nodes are arguments and an arrow from a node $A$ to a node $B$ represents an attack from argument $A$ against $B$.

\begin{remark}
	Later in this dissertation assumptions that need to hold for a specific model are introduced. They are as well represented as a directed graph, but here an edge from assumption $A$ to model $B$ denotes that the assumption $A$ needs to hold so that $B$ is a possible model. However, the difference will be determinable from the context.
\end{remark}


\begin{exa}
 \autoref{fig:small_af} represents an \gls{AF} with the definition $AF = \langle \{A, B, C, D, E\}, \allowbreak \{(A,B), (B, C),\allowbreak (B, E), (C, B),\allowbreak (D,C), (D,D)\} \rangle $. This framework will be used as an example for the following definitions.
\end{exa}


\begin{figure}[h]
\centering
\begin{tikzpicture}[->,>=stealth',shorten >=1pt,auto,node distance=2cm,
                    thick]

  \node[main ] (A) {A};
  \node[main ] (B) [right of=A] {B};
  \node[main ] (C) [right of=B] {C};
  \node[main ] (D) [below of=C] {D};
  \node[main ] (E) [below of=B] {E};
  
  \path[every node/.style={font=\sffamily\small}]
    (A) edge node [left] {} (B)
    (B) edge [bend right] node [left] {} (C)
    	edge node [left] {} (E)
    (C) edge [bend right] node [left] {} (B)
    (D) edge node [left] {} (C)
        edge [loop right] node {} (D)
    ;
\end{tikzpicture}
\caption{Small example argumentation framework.}
\label{fig:small_af}
\end{figure}


\begin{notation}
In this paper we use capital letters $\{A, B, ...\}$ to denote arguments. $AB$ or $(A, B)$ denotes an attack from $A$ to $B$ ($(A,B) \in \R$).	
\end{notation}

For the following definitions let $AF=\langle \S, \R \rangle, S' \subseteq \S$.

\begin{definition}
	A subset $S'$ is \textbf{conflict-free} iff $ \forall A, B \in S': (A, B) \notin \R$ \textit{(the subset has no attacks between its arguments)}.
\end{definition}
\begin{remark}
	Conflict-free subsets are of interest, as these sets are not directly contradictory. In other words, a conflict-free subset of an argumentation framework does not contain any attacks between its members.
\end{remark}
\begin{exa}
Conflict-free subsets in the provided example are $\emptyset, \{A\}, \{B\}, \{C, E\}, ...$. The sets $\{D\}, \{A, B\}, \{B, C\}$ are not conflict-free.
\end{exa}

\begin{definition}
$A$ is \textbf{acceptable} w.r.t. a subset $S'$ iff  $\forall B \in \S: (B, A) \in \R \Rightarrow \exists C \in S': (C, B) \in \R$ \textit{(an argument $A$ is acceptable with respect to a subset $S'$ iff for each attacker $C$ of $A$ there is an argument in $S'$ that attacks this attacker of $A$). }
\end{definition}

\begin{remark}
If an argument $A$ is acceptable w.r.t. a subset $S'$, then there exists no holding counter-argument in this \gls{AF} causing the argument $A$ not to hold.
\end{remark}

\begin{lemma}
	$S'$ \textbf{defends} $X$, if and only if $X$ is acceptable with respect to $S'$.
\end{lemma}

\begin{exa}
In  \autoref{fig:small_af} $E$ is acceptable with respect to $\{A\}$. $A$ is acceptable with respect to $\emptyset$. 
\end{exa}


\begin{definition}
The \textbf{characteristic function} of an argumentation framework $AF$ (denoted as $F_{AF}$) is defined by the following:
	\begin{align*}
		&F_{AF}: 2^{\S} \rightarrow 2^\S &\\
		&F_{AF}(S') = \{A | A~\text{is acceptable with respect to}~S'\} &
	\end{align*}
\end{definition}
\begin{notation}
	If it is unambiguous, we often refer to $F_{AF}$ with $F$. 
\end{notation}

\begin{definition}
	A conflict-free set $S' \subseteq \S$ is \textbf{admissible} iff $S'$ defends all of its arguments (\textit{each argument in $S'$ is acceptable with respect to $S'$}) or $S' \subseteq F_{AF}(S')$.
\end{definition}

\begin{exa}
In the given example in \autoref{fig:small_af} sets $\{A\}, \{A, E\}$ are admissible, and $\{B\}$ is conflict-free but not admissible, since $B$ is not acceptable with respect to $\{B\}$.
\end{exa}

\begin{definition}
$S'$ is a \textbf{complete extension} iff  $S'$ is admissible and each argument that is acceptable with respect to $S'$ (which is defended by $S'$) belongs to $S'$.
\end{definition}

\begin{remark}
	A complete extension is a set of arguments that defends all members and includes all arguments that can be accepted regarding these members. In \autoref{fig:small_af} the set $S'=\{A\}$ is acceptable, but as this set defends as well $E$ (the only attacker $B$ is not acceptable w.r.t. $S'$), this argument must be included in $S'$ to make the set complete. Hence $S' = \{A, E\}$ is a complete extension. 
\end{remark}


\begin{exa}
$X = \{A, E\}$ is a complete extension in \autoref{fig:small_af}, since it is admissible and it defends $A$ and $E$. Note that $C$ is not defended by $X$ since it does not attack $D$ and $D$ cannot be defended.
Complete extensions in \autoref{fig:pref_af} are $\{A\}, \{A, C\}$ and $\{A, D\}$.
\end{exa}


\begin{definition}
A grounded extension ($GE_{AF}$) of an argumentation framework $AF$ is the minimal (with respect to set inclusion) complete extension of $AF$. In other words $GE_{AF}$ is the least fixed point of $F_{AF}$. ($GE_{AF} = F_{AF}(\emptyset)$).
\end{definition}

\begin{remark}
	The grounded extension can be understood as the set of arguments an rational agent can accept without doubts, as it contains only the minimal acceptable arguments for an argumentation framework and does not require the agent to assume anything about any argument.
\end{remark}

\begin{exa}
	The grounded extension of the example framework in \autoref{fig:small_af} is $\{A, E\}$.
	The grounded extension of the example framework in \autoref{fig:pref_af} is $\{A\}$.
\end{exa}


\begin{definition}
\label{def:preferred_extension}
A preferred extension of an argumentation framework $AF$ is a maximised (with respect to set inclusion) admissible set of $AF$.
\end{definition}

\begin{figure}[t]
\centering
\begin{tikzpicture}[->,>=stealth',shorten >=1pt,auto,node distance=2cm,
                    thick]

  \node[main] (A) {A};
  \node[main] (B) [right of=A] {B};
  \node[main] (C) [right of=B] {C};
  \node[main] (D) [right of=C] {D};
  \node[main] (E) [right of=D] {E};
  
  \path[every node/.style={font=\sffamily\small}]
    (A) edge node [left] {} (B)
    (C) edge node [left] {} (B)
    	edge [bend right] node [left] {} (D)
    (D) edge [bend right] node [left] {} (C)
	    edge [] node [left] {} (E)
    (E) edge [loop right] node {} (E)
    ;
\end{tikzpicture}
\caption{Argumentation framework used for some examples. }
\label{fig:pref_af}
\end{figure}

\begin{exa}
The preferred extensions in \autoref{fig:pref_af} are $\{A, C\}$ and $\{A, D\}$.
\end{exa}

\begin{definition}
A conflict-free set of arguments $S' \subseteq \S$ is called a \textbf{stable extension} iff $S'$ directly attacks each argument which does not belong to $S'$.
\end{definition}
\begin{exa}
The $AF$ in \autoref{fig:pref_af} has the stable extension $\{A, D\}$. $\{A, C\}$ is a preferred, but not a stable extension.
\end{exa}


\begin{remark}
The different extensions of a $AF = \langle \S, \R \rangle$ have the following relations between each other \cite{dung1995}:
\begin{itemize} 
	\item Each preferred extension is as well a complete extension.
	\item Each stable extension is as well a preferred extension.
	\item The grounded extension is the least (with respect to set inclusion) complete extension and therefore unique for each $AF$.
	\item Preferred extensions are the most (with respect to set inclusion) complete extensions.
	\item Arguments accepted in the grounded extension are skeptically accepted in the $AF$.
	\item Every $AF$ has at least one preferred extension.
	\item A stable extension does not always exist.
\end{itemize}
\end{remark}


\begin{definition}	
	An argument is regarded as \textbf{sceptically accepted under a semantic}, iff it is accepted in all extensions of this semantic (complete, grounded, preferred, stable). An argument is regarded as \textbf{credulously accepted under a semantic}, iff it accepted in at least one, but not all, extensions of this semantic. 	
\end{definition}

\begin{exa}
	In \autoref{fig:pref_af} $\{A\}$ is sceptically accepted under each semantic. $\{C, D\}$ are credulously accepted under a preferred semantic. $\{D\}$ is as well accepted sceptically under the stable semantic.
\end{exa}


Furthermore \cite{dung1995} introduces \textit{well-founded argumentation frameworks} (having exactly one extension which is grounded, preferred and stable), \textit{uncontroversial argumentation frameworks} and \textit{coherent argumentation frameworks} (each preferred extension of an $AF$ is stable). Regarding the problem we are looking at, defeasible argumentation is really important, as we are dealing with inconsistent \glspl{SKB}. Hence these restricted argumentation frameworks will not be used in this project, therefore they are not discussed any further.


In addition to the discussed extension-based semantics, \cite{liao} introduces a so called \textit{labelling-based approach} where there are usually three labels: \texttt{IN} (accepted argument), \texttt{OUT} (rejected argument) and \texttt{UNDEC} (undecided whether this argument is accepted or rejected) and a labelling function $\lambda: \S \rightarrow \{IN, OUT, UNDEC\}$.

By defining legally labelled arguments, definitions for \textit{conflict-free labelings},~\textit{admissible labelings} and \textit{complete/grounded/preferred labelings} can be derived. As there exists a bijective projection, they can be easily transferred to the extension-based semantics already introduced in this section. This labelling based approach has been used to implement the solving algorithm described in \cite{Modgil2009Labellings} later in this thesis (see \autoref{sub:eaf_algorithm}).



\newacronym{EAF}{EAF}{Extended argumentation framework}
\glsreset{EAF}
\subsection{\gls{EAF} - Working with Preferences in $AF$}

A Dung's argumentation framework is based on logical theory which is transformed to arguments and a binary attack relation. By applying the different extensions on the created argumentation framework accepted arguments can be evaluated. \todo{"This approach does not consider, that for practical reasons and in some applications only one unique set of accepted arguments has to be determined, where personal preferences, contextual requirements or additional information might result in some arguments having a higher priority compared to others.": ok, but this sentence is quite long.  Consider splitting it.
}This approach does not consider, that for practical reasons and in some applications only one unique set of accepted arguments has to be determined, where personal preferences, contextual requirements or additional information might result in some arguments having a higher priority compared to others. This order of preferences is often itself defeasible and conflicting and therefore subject to argumentation \cite{Modgil2009}.



In this thesis the \gls{EAF} approach introduced by Modgil in \cite{Modgil2009} will be used, as it provides a useful meta-level on preferences between other arguments, by extending Dung's framework with a new attack relation between arguments and attacks. This overcomes the issues of defining orders over preferences (see \autoref{sub:paf}) or value-evaluation functions (see \autoref{sub:vaf}) and enables us to argue about preferences regardless whether they are defeasible or conflicting. In addition it provides a really user-friendly way to consider preferences which will improve the understandability for clinicians, who will be the main actors (see \autoref{sub:sassoon:actors}) in the final system.
\autoref{fig:eaf_intro}, which is taken from \cite{Modgil2009}, shows an example for a \gls{EAF} representing the following arguments:

\begin{itemize}
	\item $A$: "Today will be dry in London since the BBC forecast sunshine"
	\item $B$: "Today will be wet in London since CNN forecast rain"
	\item $C$: "But I think the BBC are more trustworthy than CNN"
	\item $D$: "However, statistically CNN are more accurate forecasters than the BBC"
	\item $E$: "Basing on a comparison on statistics is more rigorous and rational than basing a comparison on your instincts about their relative trustworthiness"
\end{itemize}

\begin{figure}[b!]
\centering
\begin{tikzpicture}[->,>=stealth',shorten >=1pt,auto,node distance=2cm,
                    thick]

  \node[main node] (A) {A};
  \node[small,n_fill_green] (B) [left of=A] {B};
  \node[small,n_fill_green] (C) [above right of=A] {C};
  \node[main node] (D) [below right of=A] {D};
  \node[small,n_fill_green] (E) [right=3cm of A] {E};
  
  \path[every node/.style={font=\sffamily\small}]
    (A) edge [bend right, dashed] coordinate [pos=0.5] (AB) node [left] {} (B)
    (B) edge [bend right] coordinate [pos=0.5] (BA) node [left] {} (A)
    (C) edge [bend right] coordinate [pos=0.5] (CD) node [left] {} (D)
    (D) edge [bend right, dashed] coordinate [pos=0.5] (DC) node [left] {} (C)
    ;
   \path [every node/.style={red}] 
   (C) edge [->>, bend right] node [left] {} (AB)
   (D) edge [->>, bend left, dashed] node [left] {} (BA)
   (E) edge [->>] node [left] {} (DC)
   ;
\end{tikzpicture}
\caption{\gls{EAF} about the weather forcasts with preferences over arguments. Dashed attacks are canceled out. Double-arrow-headed edges represent attacks on attacks. Green nodes represent accepted arguments in the unique preferred extension.}
\label{fig:eaf_intro}
\end{figure}


\glsreset{EAF}
\begin{definition}
\todo{Definition 2.15: State more explicitly in your definition that this relationship consists of attacks by arguments on other attacks.}
	An \gls{EAF} is a triple $\langle \S, \R, \D \rangle$ with $\S$ being a set of arguments and:
	\begin{itemize}
		\item $\R \subseteq \S \times \S$: attack relation.
		\item $\D \subseteq \S \times \R$: new attack relation on attacks.
		\item $\{(A, (B, C)), (A', (C, B))\} \subseteq \D \rightarrow \{(A, A'), (A', A)\} \subseteq \R$ (any arguments expressing contradictory preferences must attack each other).
	\end{itemize}
\end{definition}

\begin{remark}
\todo{Remark 2.16: OK, but it would be better to state this in your own words rather than use a quote.}
"If $A$ attacks $(B, C) \in \R$ then $A$ expresses, that $C$ is preferred to $B$. If $A'$ attacks $(C, B)$, then $A'$ expresses that $B$ is preferred to $C$. Hence \glspl{EAF} are required to conform to the constraint that any such arguments expressing contradictory preferences must attack each other."\cite{Modgil2009}.
\end{remark}

Let $\Delta = \langle \S, \R, \D \rangle$ be a \gls{EAF} and $S' \subseteq \S$ for the following definitions.

\begin{definition}
$A$ \textbf{defeats$_{S'}$} $B$ iff $(A, B) \in \R$ and $\not \exists C \in S': (C, (A, B)) \in \D$. If $A$ defeats\low{S'} $B$ and $B$ does not defeat\low{S'} $A$ then $A$ \textbf{strictly} defeats\low{S'} $B$.
\end{definition}

\begin{notation}
For the rest of the document $A \rightarrow^{S'} B$ denotes that $A$ defeats\low{S'} B and $A \nrightarrow^{S'} B$ denotes that $A$ does not defeat\low{S'} $B$.
\end{notation}


By using this definition, similar properties (e.g. conflict-free and admissible sets, acceptability of an argument, sceptically/credulously accepted arguments, extensions) as in Dung's argumentation framework can be introduced and defined.

\begin{definition}
	$S'$ is \textbf{conflict free} iff $\forall A, B \in S': (A, B) \in \R \Rightarrow (B, A) \notin \R \wedge \exists C \in S': (C, (A, B)) \in \D$ \textit{(a subset is only conflict free, if for every attack within the subset there is no counter attack and the attack itself is canceled out by an attack on this attack from an argument that is part of the subset as well)}.
\end{definition}


\begin{figure}[h]
\centering
\begin{tikzpicture}[->,>=stealth',shorten >=1pt,auto,node distance=2cm,
                    thick]

  \node[main node] (C) [above right of=A] {C};
  \node[main node] (A) [below left of=C]{A};
  \node[main node] (B) [below right of=C] {B};
  
  \path[every node/.style={font=\sffamily\small}]
    (A) edge [] coordinate [pos=0.5] (AB) node [left] {} (B)
    ;
   \path [every node/.style={red}] 
   (C) edge [->>] node [left] {} (AB)
   ;
\end{tikzpicture}
\caption{\gls{EAF} with $\langle \S, \R, \D \rangle = \langle \{A, B, C\}, \{(A, B)\}, \{(C, (A, B))\} \rangle$.}
\label{fig:eaf_small}
\end{figure}

\begin{exa}
	The set $S' = \{A, B\}$ of the \gls{EAF} in \autoref{fig:eaf_small} is not conflict-free. But the set $S' = \{A, B, C\}$ is conflict-free as $C$ attacks the attack between $A$ and $B$ and cancels it out.
\end{exa}

\begin{lemma}
Let $S'$ be a conflict-free subset of $\S$ in $\langle \S, \R, \D \rangle$. Then for any $A, B \in S'$ $A$ does not defeat\low{S'} $B$.
\end{lemma}

\begin{definition}
\todo{Definition 2.21: Add an intuitive explanation of what a "reinstatement set" is.
You should explain how the concepts discussed in 2.2 apply to your work!}
$R_{S'} = \{X_1 \rightarrow^{S'} Y_1, ..., 	X_n \rightarrow^{S'} Y_n\}$ is called a \textbf{reinstatement set} for $C \rightarrow^{S'} B$ ($C$ defeats\low{S'} $B$), iff:
\begin{itemize}
	\item $C \rightarrow^{S'} B \in R_{S'}$,
	\item $\forall_{i=1}^n X_i \in S'$,
	\item $\forall X \rightarrow^{S'} Y \in R_{S'}, \forall Y': (Y', (X, Y)) \in \D$, there is a $X' \rightarrow^{S'} Y' \in R_{S'}$.
\end{itemize}
\end{definition}

The acceptability of an argument can now be formally defined based on the reinstatement set.

\begin{definition}
$A \in \S$ is \textbf{acceptable} w.r.t.	 $S'$, iff: $\forall B: B \rightarrow^{S'} A$, there is a $C \in S': C \rightarrow^{S'} B$ and there is a reinstatement set for $C \rightarrow^{S'} B$.
\end{definition}


\begin{figure}[h]
	\centering
	\subfigure[0.3\textwidth][$A_1$ is not acceptable w.r.t $S' = \{A_1, A_2\}$]{
		\begin{tikzpicture}[->,>=stealth',shorten >=1pt,auto,node distance=1.2cm,thin]
		
		  \node[n_fill_red,small] (A1) [] {$A_1$};
		  \node[n_fill_gray,small] (A2) [right of=A1]{$A_2$};
		  \node[small] (B1) [below of=A1] {$B_1$};
		  \node[small] (B2) [below of=A2] {$B_2$};
		  \node[small] (B3) [right of=B2] {$B_3$};
		      
		  \path[every node/.style={font=\sffamily\small}]
		    (A1) edge [bend left] coordinate [pos=0.5] (A1B1) node [left] {} (B1)
		    (B1) edge [bend left] coordinate [pos=0.5] (B1A1) node [left] {} (A1)
			(A2) edge [] coordinate [pos=0.5] (A2B2) node [left] {} (B2)
		    ;
		   \path [every node/.style={red}] 
		   (B2) edge [->>] node [left] {} (A1B1)
		   (B3) edge [->>] node [left] {} (A2B2)
		   ;
		\end{tikzpicture}
		\label{fig:eaf_not_acceptable}
	}
	\hfill
	\subfigure[0.3\textwidth][$C$ is acceptable w.r.t $S' = \{A_1, A_2, A_3\}$]{
		\begin{tikzpicture}[->,>=stealth',shorten >=1pt,auto,node distance=1.2cm,
		                    thin]
		
		  \node[n_fill_gray,small] (A1) [] {$A_1$};
		  \node[n_fill_gray,small] (A2) [right of=A1]{$A_2$};
		  \node[n_fill_gray,small] (A3) [right of=A2]{$A_3$};
		  \node[small] (B1) [below of=A1] {$B_1$};
		  \node[small] (B2) [below of=A2] {$B_2$};
		  \node[small] (B3) [below of=A3] {$B_3$};
		  \node[n_fill_green,small] (C) [left of=B1] {$C$};      
		
		  \path[every node/.style={font=\sffamily\small}]
		    (A1) edge [] coordinate [pos=0.5] (A1B1) node [left] {} (B1)
		    (A2) edge [] coordinate [pos=0.5] (A2B2) node [left] {} (B2)
		    (A3) edge [] coordinate [pos=0.5] (A3B3) node [left] {} (B3)
			(A2) edge [] coordinate [pos=0.5] (A2B2) node [left] {} (B2)
			(B1) edge [] coordinate [pos=0.5] (B1C) node [left] {} (C)
		    ;
		   \path [every node/.style={red}] 
		   (B2) edge [->>] node [left] {} (A1B1)
		   (B2) edge [->>] node [left] {} (A3B3)
		   (B3) edge [->>] node [left] {} (A2B2)
		   ;
		\end{tikzpicture}
		\label{fig:eaf_acceptable}
	}
	\hfill
	\subfigure[0.3\textwidth][$C$ is not acceptable w.r.t $S' = \{A_1, A_2, A_3\}$]{
		\begin{tikzpicture}[->,>=stealth',shorten >=1pt,auto,node distance=1.2cm,
		                    thin]
		
		  \node[n_fill_gray,small] (A1) [] {$A_1$};
		  \node[n_fill_gray,small] (A2) [right of=A1]{$A_2$};
		  \node[n_fill_gray,small] (A3) [right of=A2]{$A_3$};
		  \node[small] (B1) [below of=A1] {$B_1$};
		  \node[small] (B2) [below of=A2] {$B_2$};
		  \node[small] (B3) [below of=A3] {$B_3$};
		  \node[small] (B4) [right of=B3] {$B_4$};
		  \node[n_fill_red,small] (C) [left of=B1] {$C$};      
		
		  \path[every node/.style={font=\sffamily\small}]
		    (A1) edge [] coordinate [pos=0.5] (A1B1) node [left] {} (B1)
		    (A2) edge [] coordinate [pos=0.5] (A2B2) node [left] {} (B2)
		    (A3) edge [] coordinate [pos=0.5] (A3B3) node [left] {} (B3)
			(A2) edge [] coordinate [pos=0.5] (A2B2) node [left] {} (B2)
			(B1) edge [] coordinate [pos=0.5] (B1C) node [left] {} (C)
		    ;
		   \path [every node/.style={red}] 
		   (B2) edge [->>] node [left] {} (A1B1)
		   (B2) edge [->>] node [left] {} (A3B3)
		   (B3) edge [->>] node [left] {} (A2B2)
		   (B4) edge [->>] node [left] {} (A3B3)
		   ;
		\end{tikzpicture}
		\label{fig:eaf_not_acceptable_big}
	}
	\caption{\glspl{EAF} with acceptable and not acceptable sets $S'$. $A_x$ are elements of $S'$.}
\end{figure}

\begin{exa}
	In Figure \autoref{fig:eaf_not_acceptable} $S'=\{A_1, A_2\}$ is not admissible since $A_1$ is not acceptable w.r.t. $S'$. In Figure \autoref{fig:eaf_acceptable} $C$ is acceptable w.r.t $S' = \{A_1, A_2, A_3\}$ as there is a reinstatement set $\{A_1 \rightarrow^{S'} B_1, A_2 \rightarrow^{S'} B_2, A_3 \rightarrow^{S'} B_3\}$ for $A_1 \rightarrow^{S'} B_1$. In Figure \autoref{fig:eaf_not_acceptable_big} there is an additional argument $B_4$ such that $B_4 \rightarrow (C_3 \rightarrow B_3)$ and no argument in $S'$ that defeats\low{S'} $B_4$, then no reinstatement set for $A_1 \rightarrow^{S'} B_1$ would exist, hence $C$ is not acceptable w.r.t. $S'$.
\end{exa}

Similar to Dung's theory, admissible, preferred, complete and stable extensions of an \gls{EAF} can now be defined.


\begin{definition}
	Let $S'$ be a \textbf{conflict free} subset of $\S$ in $\langle \S, \R, \D \rangle$. Then:
	\begin{itemize}
		\item $S'$ is an \textbf{admissible} extension iff every argument in $S'$ is acceptable w.r.t $S'$.
		\item $S'$ is a \textbf{preferred} extension iff $S'$ is (w.r.t. set inclusion) a maximal admissible extension.
		\item $S'$ is a \textbf{complete} extension iff each argument which is acceptable w.r.t. $S'$ is in $S'$.
		\item $S'$ is a \textbf{stable} extension iff $\forall B \notin S', \exists A \in S'$ such that $A$ defeats\low{S'} $B$.
	\end{itemize}
\end{definition}


By using this definition, we can define again \textbf{sceptically}, respectively \textbf{credulously}, accepted arguments under the semantic $ s \in $\{preferred, complete, stable\} iff $A$ is in every (at least one) $s$ extension. 
\begin{exa}
	The example given in \autoref{fig:eaf_intro} has only the single preferred, complete and stable extension $\{B, C, E\}$. \autoref{fig:eaf_small} has the admissible sets $\{A\}, \{A, C\}, \{A, B, C\}$. $\{A, B, C\}$ is the only preferred extension which is as well stable.
\end{exa}

\begin{lemma}
	Let $\Delta = \langle \S, \R, \D \rangle$ be an \gls{EAF}, $S'$ an admissible extension of $\Delta$ and let $A, A'$ be arguments which are acceptable w.r.t. $S'$. Then:
	\begin{itemize}
		\item $S'' = S' \cup \{A\}$ is admissible.
		\item $A'$ is acceptable w.r.t. $S''$.
	\end{itemize}	
\end{lemma}

\begin{lemma}
	Let $\Delta = \langle \S, \R, \D \rangle$ be an \gls{EAF}. 
	\begin{itemize}
		\item The set of all admissible extensions of $\Delta$ form a complete partial order w.r.t. set inclusion.
		\item For each admissible extension $E$ of $\Delta$ there exists a preferred extension $E'$ such that $E\subseteq E'$.
	\end{itemize}
	\label{lem:eaf:partialorder}
\end{lemma}

The definition of the characteristic function for an \gls{EAF} is similar but not equal to Dung's definition. 

\begin{definition}
	Let $\Delta = \langle \S, \R, \D \rangle$ be an \gls{EAF}, $S'\subseteq \S$, and $2^{\S_C}$ denote the set of all conflict free subsets of $\S$. The \textbf{characteristic function} $F_\Delta$ of $\Delta$ is defined as follows:
	\begin{itemize}
		\item $F_\Delta: 2^{\S_C} \rightarrow 2^{\S}$
		\item $F_\Delta(S') = \{A | A~\text{is acceptable w.r.t.~}S'\}$.
	\end{itemize}
\end{definition}

From here on we will always refer to a fixed \gls{EAF}, hence we can simply write $F$ rather than $F_\Delta$. Equally to Dung's Framework, any conflict-free set $S' \subseteq \S$ in $\Delta$ is admissible iff $S' \subseteq F(S')$, and complete iff $S'$ is a fixed point of $F$. We can apply $F$ iteratively on an \gls{EAF}: $F^0 = \emptyset, F^{i+1} = F(F^i))$. Note, that for \glspl{EAF} the characteristic function $F$ is in general \textbf{not} monotonic (e.g. $C$ is acceptable w.r.t $S'= \{A_1, A_2, A_3\}$ in \autoref{fig:eaf_acceptable}, but is not acceptable w.r.t. the conflict-free $S'' = S' \cup \{B_2, B_3\}$).
\begin{lemma}
Let $F$ be the characteristic function of an \gls{EAF}, and $F^0 = \emptyset, F^{i+1} = F(F^i)$. Then $\forall i, F^i \subseteq F^{i+1}$ and $F^i$ is conflict free.
\end{lemma}


\begin{definition}
	$\Delta = \langle \S, \R, \D \rangle$ is a \textbf{finitary} \gls{EAF} iff $\forall A \in \S$, the set $\{B | (B, A) \in \R\}$ is finite and $\forall (A, B) \in \R$, the set $\{C | (C, (A, B))\in \D\}$ is finite.
\end{definition}

\begin{definition}
	Let $\Delta$ be a finitary \gls{EAF} and $F^0 = \emptyset, F^{i+1} = F(F^i)$. Then $\cup_{i=0}^\infty(F^i)$ is the \textbf{grounded extension} $GE(\Delta)$ of $\Delta$.
\end{definition}


\begin{remark}
Similar to Dung's framework we can state the following relations between different extensions:
\begin{itemize}
	\item Every \gls{EAF} has at least one preferred extension (implied by \autoref{lem:eaf:partialorder} as $\emptyset$ is an admissible extension for every \gls{EAF}).
	\item Every stable extension of an \gls{EAF} is a preferred extension.
	\item The grounded extension for \gls{EAF} is not defined over the least fix point of the characteristic function $F$, but can be defined for finitary \glspl{EAF} over the union of all characteristic functions $F^i$.
\end{itemize}	
\end{remark}

In addition Modgil \textit{et al.} introduce in \cite{Modgil2009} the concept of a special class of \glspl{EAF}, \textit{hiearchical \glspl{EAF}}, which are defined by the existence of a partition of $\Delta$ into multiple regular Dung argumentation frameworks. Due to this restriction it is possible to define a least fix point of the characteristic function $F$, hence defining the grounded extension $GE(\Delta)$ in the same way as it has been defined in Dung's framework.

Furthermore \cite{Modgil2009} presents as well the concept of \textbf{Preference symmetric extended argumentation frameworks}. This extension limits the attacks on attacks (so the set $\D$) only on attacks between symmetrically attacking arguments. However, this is a to narrow restriction for our purposes, therefore this will not be described here. 



\subsection{Other Approaches for Preferences in Argumentation Frameworks}
\label{app:other_afs}
\label{sub:paf}
Other approaches to deal with preferences in argumentation frameworks have been proposed by \cite{amgoud,amgoud1998,Bench2003,pollock1987, prakken1997}. A \gls{PAF} defined in \cite{amgoud1998} is  a triple $\langle \S, \R, Pref \rangle$ where $Pref$ is a (partial of complete) order preordering on $\S \times \S$. The difference between this approach and the one we use is mostly the requirement of a strict ordering that has to be associated with $Pref$ and must be explicitly defined.

Modgil \textit{et al.} introduce in \cite{Modgil2009} the concept of a special class of \textit{hiearchical \glspl{EAF}}, which are defined by the existence of a partition of $\Delta$ into multiple regular Dung argumentation frameworks. Due to this restriction it is possible to define a least fix point of the characteristic function $F$, hence defining the grounded extension $GE(\Delta)$ in the same way, as it has been defined in Dung's framework. However, this is a to narrow restriction for our purposes, therefore this will not be described here. 

\label{sub:vaf}

\Glspl{VAF} as proposed in \cite{Bench2003} define an argumentation framework as a 5-tuple: $\langle \S, \R, V, val, P \rangle$ ($\S$: Arguments, $\R$: Attack relation, $V$: nonempty set of values, $val(\cdot): \S \rightarrow V$: value mapping function, $P$: set of audiences $\{a_1, ..., a_n\}$ where each audience names a total ordering $>_{a_i}$ on $V \times V$). By referring to one specific audience we retrieve a \gls{aVAF}. The set of audiences $P$ is introduced to be able to make use of preferences between values in $V$, so we might have as many audiences as there are orderings on $V$. The new definition of an argument that defeats another argument takes into account the audience $a$ and the $val(\cdot)$ of both arguments to define successful attacks. This approach requires a value mapping function $val(\cdot)$ and doesn't argue with preferences in a natural way. 

However, \cite{Modgil2009} proves that an \gls{aVAF} can be transferred to an \textit{equivalent} \gls{EAF} by representing the \gls{aVAF} with three layers: The outer layer expressing the audience, the second layer expressing the pairwise orderings on values in $V$, and the inner layer based on the actual arguments and attacks in the \gls{aVAF}.

Defeasible reasoning and preferences and their impact on argumentation frameworks are formalised as logical formalism by \cite{pollock1987, prakken1997} in the underlying logical formalism which will be used to instantiate a regular Dung framework.
\subsection{Related source paper}
The foundation of this MSc Project is the paper \cite{sassoon2014}. The goal of this project is to use Argumentation Theory for Statistical Model Selection in mostly clinical environments. The increasing availability and the growing size of datasets available for clinicians and the raising awareness on evidence based decision making extend the demand into systems that support clinicians in the analysis of this data in their day-to-day practice. In the following section a short summary on the paper will be given. More details can be found in the following parts of the paper.

To overcome the issue of clinicians analysing data and models on there compatibility \cite{sassoon2014} proposes an approach of an intelligent model selection system which is capable of suggesting appropriate model(s) to a clinician during the design stage of a study, taking in account the research question, the clinical data and any external relevant input and preferences. In addition this system should be able of supporting its decision by providing the argumentation for and against a model to the user. As clinicians might not always be qualified to perform the statistical analysis required for their research question, the process of designing models, specifying its requirements and providing the arguments for or agains these models should be separated from the actual design process and be done by a statistician. The statistician is thereby  charge of understanding the data in the context of the research question and provide arguments that are able to recommend the best suited statistical analysis for a particular research question. 

Furthermore  Sassoon \textit{et al.} \cite{sassoon2014} addresses the problem of defeasible knowledge, as the (counter)arguments for a model are often contradicting and the system, the clinician and the statistician might have some preferences for one or the other model. Therefore Sassoon \textit{et al.} propose to split the problem into two parts (i) a (defeasible) \textit{knowledge base} that contains the statistical model definitions, the objectives and assumptions of a model; (ii) \textit{argumentation schemes} to guide the model selection process and to represent expressed preferences.

The \textit{knowledge base} is used to instantiate the \textit{argumentation schemes}. The knowledge base itself defines how research objectives can be achieved through different statistical models considering their given assumptions. \textit{Research objectives} are defined as different 'families' of analysis (e.g. survival analysis or categorical outcome variable analysis).


\newacronym{SKB}{SKB}{statistical knowledge base}

\glsreset{SKB}
\subsubsection*{\Gls{SKB}}


The \gls{SKB} consists of objectives $O=\{o_1, ..., o_u\}$, models $M=\{m_1, ..., m_v\}$ and assumptions $A = \{a_1, ..., a_w\}$. Objectives correspond to different types of research questions answerable by means of statistical analysis techniques. Models represent statistical analysis techniques employable to answer a research question. Assumptions are conditions that ought to be met to employ a model.
\begin{definition}
	Let $R_{OM}: O \times M$ be a relationship such that $(o_i, m_j)\in R_{OM}$ implies that objective $o_i$ can be achieved by means of model $m_j$. Each objective can be achieved by one or more models, each model can answer one or more objectives.
\end{definition}

\begin{definition}
	Let $R_{MA}: M \times A$ be relation between models and their assumptions. $(m_i, a_j)\in R_{MA}$ implies that the model $m_i$ requires the assumption $a_j$ to be true to be applyable. Each model has a set of assumptions that must be met adequately if a model is applied. Let $A(m_i) = \{a_j | (m_i, a_j) \in R_{MA}\}$ be the set of assumptions of $m_i$.
\end{definition}

As it will not always be possible to find a model where all assumptions are met, it will be necessary to apply a model even when there are some violations regarding the assumptions. Hence we need to specify whether an assumption $a_j$ is \textit{critical} to a model. If a critical assumption doesn't hold its model must not be applied under any circumstances.

\begin{definition}
Let $C \subset R_{MA}$ be the set of critical assumptions. To apply a model $m_i$ all critical assumptions $A_c(m_i) = \{a_j | (m_i, a_j) \in C\}$ must be met.
\end{definition}

Each \textit{assumption} will be either specified as a specific property of the data set (assessed by applying tests on the data set) or as a characteristic of the population of interest or the way in which the data set was collected from that population (relying on the expertise of a domain expert). Sassoon proposes therefore a partitioning of all assumptions: $A_t$ denotes the set of tests (and apply a test on the available data set). $A_q$ denotes the set of queries (and will be assessed by asking the clinician for an opinion). Lets define $A_t(m_i)= \{a_j| (m_i, a_j) \in A_t\}$ and $A_q(m_i)= \{a_j| (m_i, a_j) \in A_q\}$. Note that critical assumptions can be in either $A_t$ or $A_q$: $C \subset R_{MA} = A_q \cup A_t, ~A_q \cup A_t = \emptyset$. 

The \autoref{fig:skb_example} shows the structure between \textit{objectives}, \textit{models} and \textit{assumptions} in a small example. The assumptions $\{a_1, a_2, a_3\} \in A_t$ are based on the provided data set, $\{a_3, a_5\} \in A_q$ are based on the domain expertise. The objectives $\{o_1, o_2\}$ can both be achieved by the possible model $m_1$, $o_3$ can only be achieved by the possible model $m_3$. Note that $m_3$ is still a possible model although the non critical assumption $a_4$ doesn't hold.

\begin{figure}[h]
\centering
\begin{tikzpicture}[->,>=stealth',shorten >=1pt,auto,node distance=1cm,
                    thick]

  \node[main node] (o1) {$o_1$};
  \node[main node] (o2) [below of=o1] {$o_2$};
  \node[main node] (o3) [below of=o2] {$o_3$};

  \node[main node,n_fill_green] (m1) [right= 1.5cm of o1] {$m_1$};
  \node[main node,n_fill_red] (m2) [below of=m1] {$m_2$};
  \node[main node,n_fill_green] (m3) [below of=m2] {$m_3$};

  \node[main node, double,n_fill_red] (a2) [right= 1.5cm of m1] {$a_2$};  
  \node[main node,n_fill_green] (a1) [above of=a2] {$a_1$};
  \node[main node, double,n_fill_green] (a3) [below of=a2] {$a_3$};
  \node[main node,n_fill_red] (a4) [below of=a3] {$a_4$};
  \node[main node, double,n_fill_green] (a5) [below of=a4] {$a_5$};
  
  \node (DB) [cylinder, shape border rotate=90,draw,minimum height=2cm,minimum width=1.3cm, fill=gray!10][right= 3cm of a2]{DB};
  \node (T) [draw=black,fill=gray!10,cloud,font=\fontfamily{ppl}\fontsize{1cm}{1.5cm}\selectfont][right= 3cm of a4]{?};
  
  \path[every node/.style={font=\sffamily\small}]
    (a1) edge [] node [left] {} (m1)
         edge [] node [left] {} (m2)
    (a2) edge [] node [left] {} (m2)
    (a3) edge [] node [left] {} (m2)
    (a4) edge [] node [left] {} (m2)
         edge [dashed] node [left] {} (m3)
         edge [] node [left] {} (m3)
    (a5) edge [] node [left] {} (m3)
    (m1) edge [] node [left] {} (o1)
         edge [] node [left] {} (o2)
	(m2) edge [] node [left] {} (o1)
	     edge [] node [left] {} (o2)
	(m3) edge [] node [left] {} (o3)
	(DB) edge [] node [left] {} (a1)
         edge [] node [left] {} (a2)
         edge [] node [left] {} (a4)
         edge [] node [left] {} (a1)
    (T)  edge [] node [left] {} (a3)
         edge [] node [left] {} (a5)
    ;
\end{tikzpicture}
\caption{Example of a \gls{SKB}. Double circled assumptions represent critical assumptions. Green coloured assumptions hold while red coloured assumptions do not hold. Green coloured models are the possible models.}
\label{fig:skb_example}
\end{figure}



\subsubsection*{Different processes to instantiate arguments from the knowledge base}

To achieve an objective $o_c$ (which has been selected by the clinician) a number of models $m_i$ might be possible providing their critical assumptions $A_c(m_i)$ are met. The process of instantiating the arguments can be seen in AS\autoref{as:1}. Note that there might be multiple possible models evaluated by AS\ref{as:1} to be acceptable. To elaborate one model $m_i$ which is preferred over the other possible models we define a Model Preference as depicted in AS\ref{as:2}.

\begin{AS}[h]
\centering
	\caption{Constructed argument for a Possible Model.\label{as:1}}
	\fbox{\begin{minipage}{0.7\textwidth}
	\begin{itemize}
		\item Model $m_i$ achieves objective $o_c$.
		\item The data set meets the set of assumptions $A_t' = A_t(m_i)$.
		\item The research project meets the set of assumptions $A_q' = A_q(m_i)$.
		\item $A_c(m_i) \subseteq A_t' \cup A_q'$.
	\end{itemize}
	\rule{\textwidth}{0.5pt}\\
	$~~~~\Rightarrow$  $m_i$ is a possible model for the research question $o_c$.
	\end{minipage}}
\end{AS}

Depending on the purposes of models and on the reasons for preferring one model over another there are different ways of implementing the $(\ast)$ in the generic definition provided in AS\ref{as:2}. These different reasons to prefer one model over another depend on the context and on the application, however \cite{sassoon2014} proposes different approaches to express the preference of one model over another. The essential idea of these preferences is based on the number and importance of violated assumptions or the total number of satisfied assumptions. 

\begin{AS}[h]
\centering
	\caption{Argument for a Model Preference between two possible models.\label{as:2}}
\fbox{\begin{minipage}{0.7\textwidth}
	\begin{itemize}
		\item $m_i$ is a possible model.
		\item $m_j$ is a possible model.
		\item there is some reason to prefer $m_i$ over $m_j$ ($\ast$)
	\end{itemize}
	\rule{\textwidth}{0.5pt}\\
	$~~~~\Rightarrow$  $m_i$ is preferred over $m_j$.
	\end{minipage}}

\end{AS}

To express the preferences for one model over another, Sassoon \textit{et al.} \cite{sassoon2016} propose to use \gls{EAF} \cite{Modgil2009} in combination with \gls{PAF} \cite{amgoud,amgoud2000} approaches, both been presented earlier in this paper. These modifications enable argumentation on preferences which are encapsulated as arguments, therefore we will be able to consider conflicting preferences in our system and ensure scalability. By doing so we can as well express orders of importance between statistical reasons to prefer one model over another and clinician's preferences in specific contexts.


The paper by Sassoon \textit{et al.} forms the basic requirements for the content of this project and will be regarded as specification for the project, hence the following chapters will base on it.



%%%%%%%%%%%%%%%%%%
%\todo{Maybe: Related work regarding clinical data analysis}
%\subsection{Related work regarding clinical data analysis}
%\begin{itemize}
%	\item Argumentation for Aggregating Clinical Evidence \cite{hunter}: multiple outcome indicators, checks only old date, not supposed data for recommendation, focuses on generation of arguments out of evidence
%	\item An argument-based approach to reasoning with clinical knowledge\cite{Gorogiannis20091}: Focus on simple language representing the results of clinical trials, doesn't deal with preferences of different priorities.

%	\item \cite{Atkinson2006} an approach based on \gls{VAF}.
%\end{itemize}
 
