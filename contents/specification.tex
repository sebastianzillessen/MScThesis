\section{Specification}
\label{sec:specification}

In accordance with the selected methodology of Use Case 2.0 (see \autoref{sec:design}) no in-depth specification has been written, however a reasonable amount of \glspl{use_case} have been defined and iterativly refined. These were then reviewed by the client and discussed in detail. As mentioned, the project used Trello as management tool. All \glspl{use_case} have been described on cards and \glspl{use_case_slice} have been generated out of those. They are directly reflected by the description in \cite{sassoon2014,sassoon2016, sassoon2016CD}. In general each \gls{use_case} consists of the following attributes: Name, Desired outcome, (Main-)\glspl{Actor}, Flow to achieve the desired outcome, Alternative flow (optional).


\subsection{Weak requirements}
\label{sub:weak}
In addition to the described \glspl{use_case} (see \autoref{tab:usecases}) the following weak requirements influenced the development of the final project:


\begin{itemize}
	\item The system should be designed as an interactive \textbf{web application} to ensure:
	\begin{itemize}
		\item Access for multiple users.
		\item Portability between different operating systems.
		\item No requirement of installation of software on local computers.
		\item Clinician and Statistician can share the same infrastructure without the need of working at the same place.
	\end{itemize}
	\item A decent test suite should be provided to ensure the functionality of the system.
	\item The system must be suitable for different user types (\glspl{Actor}).
	\item The system has to provide different levels of access for different user groups.
	\item The system should be able to execute test scripts as \texttt{R}-Scripts.
	\item The system should be hosted on a free accessible hosting provider.
\end{itemize}


\subsection{Use Cases}
\label{sub:use_cases}
\autoref{tab:usecases} shows an overview over the major \glspl{use_case} that have been generated during the requirements analysis. For the sake of simplicity they have not been integrated into this dissertation. However they are public available on the \href{https://trello.com/b/ywCkicpc}{Trello-Board}\footnote{\url{https://trello.com/b/ywCkicpc}}.

\begin{landscape}
	\begin{longtable}{ l l p{10.5cm} l l p{3cm} }
		\textbf{ID}                         & \textbf{\Gls{Actor}}\footnote{C: Clinician, S: Statistician, A: Admin} &\textbf{Description} &  \textbf{RC} & \textbf{Prio}\footnote{MoSCoW priorities} &  \textbf{Comments}\\
		\href{https://trello.com/c/KEOokZp9}{UC1}   & 	- & 	Enter two default models into the system as code without user interface including definition of assumptions & RC1 & Must &  For initial testing purposes  \\
		\href{https://trello.com/c/ebVrFdA5}{UC2}   & 	C & 	Get models that are appropriate for a research question & RC1	& Must &	   \\
		\href{https://trello.com/c/ORRBjISQ}{UC2-1} & 	C & 	Store done analysis & RC2 & Should &  see \autoref{as:1} \\
		\href{https://trello.com/c/qKLAoWRj}{UC3} 	&  	S & 	Define assumptions for models & - & - & obsolet \\
		\href{https://trello.com/c/7NINsfz8}{UC4}   & 	C & 	Check that a dataset meets the set of assumptions of a model and that the queries to the clinician regarding the research question's  assumptions are met as well. & RC1 & Must   &   \\
		\href{https://trello.com/c/22JGne3r}{UC5}   & 	C & 	Select a research question & RC1 & Must &   \\
		\href{https://trello.com/c/CVGBVWID}{UC6}   &   C, S, A & 	Login into the system and be authenticated with your role & RC2 & Should &	 \\
		\href{https://trello.com/c/pId27kJM}{UC6-1} &   A & 	Create new users & RC2 & Must &	 \\
		\href{https://trello.com/c/pQ98qgSL}{UC6-2} &  	A & 	Delete users & RC2 & Must	 &   \\
		\href{https://trello.com/c/mvxBeNSR}{UC6-3} &  	A & 	Modify users & RC2 & Should &   \\
		\href{https://trello.com/c/DidVQKAS}{UC7}   &   C & 	Fill patient data into the system to be checked by the models & RC2 & Must &  \\
		\href{https://trello.com/c/be2088JH}{UC8}	& 	C & 	See arguments for and agains models for a research question & RC2 & Must &   \\
		\href{https://trello.com/c/Ca9mA3uA}{UC9}   &   C & 	Define new personal preferences & RC3 & Must &   \\
		\href{https://trello.com/c/Ca9mA3uA}{UC9-1}   &   S & 	Define new global preferences & RC3 & Must &   \\
		\href{https://trello.com/c/1s656fA9}{UC10}  &   C & 	Get recommended model taking into account global preferences & RC3 & Should & 	 \\
		\href{https://trello.com/c/xUDStSOK}{UC11}  &   A & 	Create new users with a specific role for the system & RC3 & Could &    \\
		\href{https://trello.com/c/5UMo7o6U}{UC12}  &   S & 	Enter additional analysis models into the system in a defined language as text so that it will be regarded as an option for applicable models	& RC3 & Should & \\
		\href{https://trello.com/c/2V6Cl65u}{UC12-1}&   S & 	Add Test-Argument into the system & RC2 & Must & \\
		\href{https://trello.com/c/OwM2Z7wt}{UC12-2}&   S & 	Add Query-Argument into the system & RC2 & Must &  \\
		\href{https://trello.com/c/VThxB5aS}{UC12-3}&   S & 	Add Test-Query-Assumptions into the system & RC4 & Should & added later\\
		\href{https://trello.com/c/CkpJUNPW}{UC12-4}&   S & 	Run Test-Argument on a specific existing dataset & RC3	& Should &  \\
		\href{https://trello.com/c/Rg6GPnNE}{UC12-5}&   S & 	Add Blank Arguments for grouping of assumptions & RC2 & Could & \\
		\href{https://trello.com/c/ORlMByiQ}{UC13}  &   S & 	Get in-depth information about algorithmic performance & RC4 & Would & rejected\\
		\href{https://trello.com/c/NcV3lo4w}{UC14}  &   C & 	Enter personal preferences for statistical models and/or analysis techniques & RC4 & Could &   \\
		\href{https://trello.com/c/BOUu2hKN}{UC15}  &   C & 	See argumentation frameworks that are generated for/agains the statistical models. Graph Export: See arguments for and agains models for a research question in a nice and graphical &RC4 & Would & 	 ignored as printing of generated graphs possible  \\
		\href{https://trello.com/c/3FCcFdmm}{UC16}  &   C & 	See analysis process of argumentation framework as visualisation & RC4 & Would	&    \\
		\href{https://trello.com/c/Hv2xe2UW}{UC17}  &   S & 	Enter additional research questions into the system & RC3 &Could& \\
		\href{https://trello.com/c/w1YiIgU7}{UC17-1}&   S & 	Modify research questions&RC3 & Must &  \\
		\href{https://trello.com/c/UbT5mtDx}{UC17-2}&   S & 	Delete research questions&RC3&Must& \\
		\href{https://trello.com/c/dpLHOxbB}{UC18}  &   S & 	Include R-script execution	 & RC1 & Must & \\
		\href{https://trello.com/c/bZHdWpkt}{UC19}  &   S & 	Enter global preferences for statistical models and/or analysis techniques& RC3 & Could &	 \\

		\caption{List of the different use cases that have been evaluated during the requirements analysis.}	
		\label{tab:usecases}
	\end{longtable}
\end{landscape}

The \glspl{use_case} defined in \autoref{tab:usecases} were then further defined, requirements deriving from it have been generated as \glspl{use_case_slice} and were implemented by applying an iterative approach with close feedback loops to the supervisor of the project and the author of \cite{sassoon2014,sassoon2016, sassoon2016CD} who has been treated as a \gls{product owner} or client. As the software was developed in a test driven approach (see \autoref{sec:tdd}) all features where implemented by designing tests first. This is reflected by the final existing test suite (see \autoref{sub:test_suit}) which contains a decent amount of \glspl{unit_test}, \glspl{integration_test} and \glspl{e2e_test} and has an overall line coverage of $ \geq 85\%$ (see \href{https://codeclimate.com/github/sebastianzillessen/small-data-analyst/coverage}{https://codeclimate.com/github/sebastianzillessen/small-data-analyst/coverage}). \autoref{uc:12-1} shows an exemplary detailed description of one \gls{use_case}.



{ \tiny
	\begin{longtable}{|p{2cm} p{11cm}|}

		\hline
			\textbf{ID} & 
				\href{https://trello.com/c/2V6Cl65u}{UC12-1}\\
			
			\hline
			\textbf{Actor} & Statistician or Admin \\
			\hline
			\textbf{Description} & 
				Add Test-Argument into the system\\
			\hline
			\textbf{Desired~outcome} & 
				- A statistician can create new arguments \newline
				- These arguments can be assigned to models \newline
				- The arguments are accessible for all users in the system \newline
		\\
		\hline
			\textbf{Flow} & 
				1.) The user is logged in as Admin/Statistician  \newline
				2.) The User clicks on "Add Test-Argument" (see \href{https://trello.com/c/OwM2Z7wt}{Query arguments}, \href{https://trello.com/c/Rg6GPnNE/39-uc12-5-add-attacks-between-arguments}{Blank arguments})\newline
				3.) The system shows a form including "Name", "Description", "Attacking (Model or Other Argument)", "Required dataset fields", "R-Code to run test" and a disabled submit button\newline
				4.) The User enters a Name\newline
				5.) The user might enter a description\newline
				6.) The User selects at least on Model or other Argument from the list of possible attacked objects and whether it is critical or non-critical to that object.\newline
				7.) The user enters a list of required dataset fields.\newline
				8.) The user enters the R-Code to test if this argument holds or not. The R-Code must return a true/false statement at the end, \texttt{True} saying the argument holds, otherwise \texttt{False}. The R-Script will get as input a variable named \texttt{tabular\_data}  and must assign the variable \texttt{result} with \texttt{true} / \texttt{false} (see \autoref{sec:r_code}). The attributes specified in "Required dataset fields" will be checked before executing the \texttt{R}-Script.\newline
				9.) The user is in charge of verifying his script against datasets. A functionality to do so will be given by selecting a dataset in the system first. The script will then be checked against this dataset.\newline
				10.) The user clicks submit.The system performs a validation of the argument.\newline
				11.) The System stores the argument in the database and redirects the user to a list of available arguments and shows a success message.
		\\
		\hline
			\textbf{Alternatives} & 
							1.a) Not logged in as Admin: No changes to arguments possible.
				\newline	4.a) No name entered: show required inline message, disable submit button.
				\newline	6.a) No attacked object selected: show required inline message, disable submit button.
				\newline	8.a) If the user does not enter a R-Script: mark as required and do not enable submit button.
				\newline	9.a) If there is a syntax error: Show syntax error and line to the user
				\newline	9.b) If the result is not boolean: Show error explaining true/false result required.
				\newline	9.c) If there has been an error during execution with the testdataset: show error.
				\newline	9.d) After successful run: Show the dataset used to test it and the result.
				\newline	10.a) If the user does not select "Submit": NO changes done.
				\newline	11.a) If the validation fails: Redirect the user back to the form page and show validation failure.
				\newline	11.b) If any error occurs,  Redirect the user back to the form page and show error explanation.
							\\
		\hline

	\label{uc:12-1}\\
	\caption{Use case 12-1: Example for a detailed description.}
\end{longtable}
}


\subsection{Actors in our System}
\label{sub:sassoon:actors}
During the requirements analysis the following actors have been identified and are described as follows:

\textbf{Clinicians} are the main users of the system and do the actual analysis of a dataset by using the predefined research questions, models and assumptions. A clinician should see all globally available research questions, models and their assumptions. In addition the preferences defined by the statistician will apply to all analyses a clinician performs. However she/he will be able to enter personal preferences between models that only apply for his analyses. An overview over her/his uses cases can be seen in \autoref{fig:usecase:clinician}.

\textbf{Statisticians} are able to enter statistical models into the system and to define global applicable preferences. To do so, a statistician can enter different types of assumptions mapped to models and preferences. These might contain query assumptions, where the clinician performing an analysis has to confirm a fact, or test assumptions that are checked against the dataset by running a \texttt{R}-script. She or he is as well allowed to see all datasets that have been uploaded and to check which analysis have been performed (see \autoref{fig:usecase:statistician}).



\textbf{Administrators} are able to create, modify and destroy all available data in the system. In addition an administrator is required to approve new users and to assign them to a role. She or he is as well capable of inviting new users into the system. An overview over her/his uses cases can be seen in \autoref{fig:usecase:admin}.



\subsection{Technical Specification}
\label{sub:technical}

As the system should be accessible by multiple users and from different departments (e.g. clinicians and statisticians), it will be developed as a web-application in \gls{RoR}\footnote{\href{http://rubyonrails.org}{http://rubyonrails.org}} and will be hosted on Heroku\footnote{\href{https://www.heroku.com}{www.heroku.com}} as this will provide an easy-to-use and easy-to-deploy environment, where our demo application can be hosted for free. To provide a \gls{CI} environment, the project code is hosted on \texttt{github}\footnote{\href{https://github.com/sebastianzillessen/small-data-analyst}{https://github.com/sebastianzillessen/small-data-analyst}} and a \texttt{Travis CI}\footnote{\href{https://travis-ci.org/sebastianzillessen/small-data-analyst}{https://travis-ci.org/sebastianzillessen/small-data-analyst} } instance has been setup that runs all the implemented tests of the project.

The used datasets are anonymised, so data protection issues are reduced to a minimum and the datasets can be hosted in the cloud. \texttt{PostgreSQL}\footnote{\href{http://www.postgresql.org}{http://www.postgresql.org}} will be used as a database, as it is well integrated on Heroku and provides high scalability.

Most of the assumption tests for the statistical models will be performed in \gls{R}, as this is a common used language for statistical calculations and well known by statisticians. In addition many assumption-checks already exist in \gls{R}. These scripts (externally provided) will be executed in the \gls{RoR} application with the help of third party gems.

To provide a responsive and clear user interface the \texttt{Bootstrap}\footnote{\href{http://getbootstrap.com}{http://getbootstrap.com}} framework will be used to design and style the application. 

\subsection{Data flow in the System}

\autoref{fig:data_flow} shows -- in a simplified representation -- the data flows in the process of statistical model selection: First of all, the statistician has to create global preferences, research questions including their possible models and the required assumptions. This is all stored in a knowledge base which is itself stored in  a persistent database. A detailed description of all database classes can be found in \autoref{sub:db}. 

\begin{figure}[!h]
\centering
\includegraphics[width=0.8\textwidth]{figures/data_flow}
\caption{Core data flow in the system to create new analyses.}
\label{fig:data_flow}
\end{figure}


A clinician can upload new clinical data retrieved from studies by using the CSV upload functionality. Starting a new analysis retrieves the dataset from the database and instantiates the knowledge base for this specific problem. The assumptions of all applicable models for the selected research question are then evaluated. If multiple models are possible, the preferences (first the globally defined ones by a statistician, later the personal preferences a clinician might have) are applied to the system one after another until one last model remains. This model is then returned as recommended model. The process of the rejection of various models will be graphically visible to the user (see \autoref{sub:ui}). The process of creating and processing a new analysis is described in detail in \autoref{sub:walk}.


