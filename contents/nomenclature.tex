\newpage
\newacronym{AI}{AI}{artificial intelligence}
\newacronym{AF}{AF}{argumentation framework}
\newacronym{EAF}{EAF}{extended argumentation framework}
\newacronym{PAF}{PAF}{preference-based argumentation framework}
\newacronym{VAF}{VAF}{value-based argumentation framework}
\newacronym{aVAF}{aVAF}{audience specific value-based argumentation framework}
\newacronym{SKB}{SKB}{statistical knowledge base}
\newacronym{RoR}{RoR}{Ruby on Rails}
\newacronym{CI}{CI}{continuous integration}
\newacronym{RQ}{RQ}{research question}
\newacronym{TDD}{TDD}{test-driven development}
\newacronym{DRY}{DRY}{don't repeat yourself}
\newacronym{UI}{UI}{user interface}
\newacronym{UML}{UML}{Unified Modelling Language}
\newacronym{MVC}{MVC}{model, view and controller}
\newacronym{KISS}{KISS}{keep it short and simple}
\newacronym[firstplural={argumentation frameworks based on contextual preferences}]{CPAF}{CPAF}{argumentation framework based on contextual preferences}




%%%% GLossaries
\newglossaryentry{research question}
{
  name={research question},
  description={An textual representation of an objective for a performed analysis in our system. Research questions have multiple models that are in general possible to analyse a certain objective}
}

\newglossaryentry{model}
{
  name={model},
  description={}
}
\newglossaryentry{assumption}
{
  name={assumption},
  description={}
}

\newglossaryentry{preference}
{
  name={preference},
  description={"Preference orders over models will arise from different sources in the context of statistical model selection" (such as the statistical theory underpinning each model, model intent and the clinicians preference) \cite{sassoon2016CD}}
}

\newglossaryentry{CD}
{
  name={context domain},
  description={Different preferences are made of sets of mutually exclusive contexts and express for each available model a performance measurement related to this context \cite{sassoon2016CD}}
}
\newglossaryentry{Actor}
{
  name={actor},
  description={Specifies a role played by a user or any other system that interacts with a use case in UML}
}

\newglossaryentry{R}
{
  name={R},
  description={"A system for statistical computation and graphics. It consists of a language plus a run-time environment with graphics, a debugger, access to certain system functions, and the ability to run programs stored in script files." It is widely used for statistical analysis processes. \cite{R}}
}
\newglossaryentry{product owner}
{
  name={product owner},
  description={"The Scrum product owner is typically a project's key stakeholder. Part of the product owner responsibilities is to have a vision of what he or she wishes to build, and convey that vision to the Scrum team. This is key to successfully starting any agile software development project." \cite{product_owner}}
}
\newglossaryentry{use_case}
{
  name={use case},
  description={"The use cases capture the goals of the system. To understand a use case we tell stories. The stories cover how to successfully achieve the goal, and how to handle any problems that may occur on the way. Use cases provide a way to identify and capture all the different but related stories in a simple but comprehensive way. This enables the system’s requirements to be easily captured, shared and understood." \cite{jacobson2011usecase}}
  }

\newglossaryentry{use_case_flow}
{
  name={use case flow},
  description={"Use-Case Narrative that outlines its stories as a set of flows" \cite{jacobson2011usecase}}
}
\newglossaryentry{use_case_slice}
{
  name={use case slice},
  description={Use cases are build up out of multiple "use case slices (a slice being a carefully selected part of a use case) assist systematically in finding the application architecture" \cite{jacobson2011usecase}}
}

\newglossaryentry{unit_test}{
	name={unit test},
	description={Involves testing the smallest testable parts of an application (unit). Unit tests can be run individually and independently and test one specific components feature set. Dependencies are stubbed or mocked out}
}
\newglossaryentry{integration_test}{
	name={integration test},
	description={ Individual software modules are combined and tested as a group to check whether a desired feature is implemented correctly. Usually two to four components are tested together to ensure that interfaces are developed in accordance to the specification}
}
\newglossaryentry{e2e_test}{
	name={end-to-end test},
	description={Tests whether the flow of an application is behaving as expected. End-to-end tests ensure data integrity and usability of an application}
}


\printglossaries

\newpage